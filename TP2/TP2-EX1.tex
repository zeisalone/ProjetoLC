\documentclass[11pt]{article}

    \usepackage[breakable]{tcolorbox}
    \usepackage{parskip} % Stop auto-indenting (to mimic markdown behaviour)
    

    % Basic figure setup, for now with no caption control since it's done
    % automatically by Pandoc (which extracts ![](path) syntax from Markdown).
    \usepackage{graphicx}
    % Maintain compatibility with old templates. Remove in nbconvert 6.0
    \let\Oldincludegraphics\includegraphics
    % Ensure that by default, figures have no caption (until we provide a
    % proper Figure object with a Caption API and a way to capture that
    % in the conversion process - todo).
    \usepackage{caption}
    \DeclareCaptionFormat{nocaption}{}
    \captionsetup{format=nocaption,aboveskip=0pt,belowskip=0pt}

    \usepackage{float}
    \floatplacement{figure}{H} % forces figures to be placed at the correct location
    \usepackage{xcolor} % Allow colors to be defined
    \usepackage{enumerate} % Needed for markdown enumerations to work
    \usepackage{geometry} % Used to adjust the document margins
    \usepackage{amsmath} % Equations
    \usepackage{amssymb} % Equations
    \usepackage{textcomp} % defines textquotesingle
    % Hack from http://tex.stackexchange.com/a/47451/13684:
    \AtBeginDocument{%
        \def\PYZsq{\textquotesingle}% Upright quotes in Pygmentized code
    }
    \usepackage{upquote} % Upright quotes for verbatim code
    \usepackage{eurosym} % defines \euro

    \usepackage{iftex}
    \ifPDFTeX
        \usepackage[T1]{fontenc}
        \IfFileExists{alphabeta.sty}{
              \usepackage{alphabeta}
          }{
              \usepackage[mathletters]{ucs}
              \usepackage[utf8x]{inputenc}
          }
    \else
        \usepackage{fontspec}
        \usepackage{unicode-math}
    \fi

    \usepackage{fancyvrb} % verbatim replacement that allows latex
    \usepackage{grffile} % extends the file name processing of package graphics 
                         % to support a larger range
    \makeatletter % fix for old versions of grffile with XeLaTeX
    \@ifpackagelater{grffile}{2019/11/01}
    {
      % Do nothing on new versions
    }
    {
      \def\Gread@@xetex#1{%
        \IfFileExists{"\Gin@base".bb}%
        {\Gread@eps{\Gin@base.bb}}%
        {\Gread@@xetex@aux#1}%
      }
    }
    \makeatother
    \usepackage[Export]{adjustbox} % Used to constrain images to a maximum size
    \adjustboxset{max size={0.9\linewidth}{0.9\paperheight}}

    % The hyperref package gives us a pdf with properly built
    % internal navigation ('pdf bookmarks' for the table of contents,
    % internal cross-reference links, web links for URLs, etc.)
    \usepackage{hyperref}
    % The default LaTeX title has an obnoxious amount of whitespace. By default,
    % titling removes some of it. It also provides customization options.
    \usepackage{titling}
    \usepackage{longtable} % longtable support required by pandoc >1.10
    \usepackage{booktabs}  % table support for pandoc > 1.12.2
    \usepackage{array}     % table support for pandoc >= 2.11.3
    \usepackage{calc}      % table minipage width calculation for pandoc >= 2.11.1
    \usepackage[inline]{enumitem} % IRkernel/repr support (it uses the enumerate* environment)
    \usepackage[normalem]{ulem} % ulem is needed to support strikethroughs (\sout)
                                % normalem makes italics be italics, not underlines
    \usepackage{mathrsfs}
    

    
    % Colors for the hyperref package
    \definecolor{urlcolor}{rgb}{0,.145,.698}
    \definecolor{linkcolor}{rgb}{.71,0.21,0.01}
    \definecolor{citecolor}{rgb}{.12,.54,.11}

    % ANSI colors
    \definecolor{ansi-black}{HTML}{3E424D}
    \definecolor{ansi-black-intense}{HTML}{282C36}
    \definecolor{ansi-red}{HTML}{E75C58}
    \definecolor{ansi-red-intense}{HTML}{B22B31}
    \definecolor{ansi-green}{HTML}{00A250}
    \definecolor{ansi-green-intense}{HTML}{007427}
    \definecolor{ansi-yellow}{HTML}{DDB62B}
    \definecolor{ansi-yellow-intense}{HTML}{B27D12}
    \definecolor{ansi-blue}{HTML}{208FFB}
    \definecolor{ansi-blue-intense}{HTML}{0065CA}
    \definecolor{ansi-magenta}{HTML}{D160C4}
    \definecolor{ansi-magenta-intense}{HTML}{A03196}
    \definecolor{ansi-cyan}{HTML}{60C6C8}
    \definecolor{ansi-cyan-intense}{HTML}{258F8F}
    \definecolor{ansi-white}{HTML}{C5C1B4}
    \definecolor{ansi-white-intense}{HTML}{A1A6B2}
    \definecolor{ansi-default-inverse-fg}{HTML}{FFFFFF}
    \definecolor{ansi-default-inverse-bg}{HTML}{000000}

    % common color for the border for error outputs.
    \definecolor{outerrorbackground}{HTML}{FFDFDF}

    % commands and environments needed by pandoc snippets
    % extracted from the output of `pandoc -s`
    \providecommand{\tightlist}{%
      \setlength{\itemsep}{0pt}\setlength{\parskip}{0pt}}
    \DefineVerbatimEnvironment{Highlighting}{Verbatim}{commandchars=\\\{\}}
    % Add ',fontsize=\small' for more characters per line
    \newenvironment{Shaded}{}{}
    \newcommand{\KeywordTok}[1]{\textcolor[rgb]{0.00,0.44,0.13}{\textbf{{#1}}}}
    \newcommand{\DataTypeTok}[1]{\textcolor[rgb]{0.56,0.13,0.00}{{#1}}}
    \newcommand{\DecValTok}[1]{\textcolor[rgb]{0.25,0.63,0.44}{{#1}}}
    \newcommand{\BaseNTok}[1]{\textcolor[rgb]{0.25,0.63,0.44}{{#1}}}
    \newcommand{\FloatTok}[1]{\textcolor[rgb]{0.25,0.63,0.44}{{#1}}}
    \newcommand{\CharTok}[1]{\textcolor[rgb]{0.25,0.44,0.63}{{#1}}}
    \newcommand{\StringTok}[1]{\textcolor[rgb]{0.25,0.44,0.63}{{#1}}}
    \newcommand{\CommentTok}[1]{\textcolor[rgb]{0.38,0.63,0.69}{\textit{{#1}}}}
    \newcommand{\OtherTok}[1]{\textcolor[rgb]{0.00,0.44,0.13}{{#1}}}
    \newcommand{\AlertTok}[1]{\textcolor[rgb]{1.00,0.00,0.00}{\textbf{{#1}}}}
    \newcommand{\FunctionTok}[1]{\textcolor[rgb]{0.02,0.16,0.49}{{#1}}}
    \newcommand{\RegionMarkerTok}[1]{{#1}}
    \newcommand{\ErrorTok}[1]{\textcolor[rgb]{1.00,0.00,0.00}{\textbf{{#1}}}}
    \newcommand{\NormalTok}[1]{{#1}}
    
    % Additional commands for more recent versions of Pandoc
    \newcommand{\ConstantTok}[1]{\textcolor[rgb]{0.53,0.00,0.00}{{#1}}}
    \newcommand{\SpecialCharTok}[1]{\textcolor[rgb]{0.25,0.44,0.63}{{#1}}}
    \newcommand{\VerbatimStringTok}[1]{\textcolor[rgb]{0.25,0.44,0.63}{{#1}}}
    \newcommand{\SpecialStringTok}[1]{\textcolor[rgb]{0.73,0.40,0.53}{{#1}}}
    \newcommand{\ImportTok}[1]{{#1}}
    \newcommand{\DocumentationTok}[1]{\textcolor[rgb]{0.73,0.13,0.13}{\textit{{#1}}}}
    \newcommand{\AnnotationTok}[1]{\textcolor[rgb]{0.38,0.63,0.69}{\textbf{\textit{{#1}}}}}
    \newcommand{\CommentVarTok}[1]{\textcolor[rgb]{0.38,0.63,0.69}{\textbf{\textit{{#1}}}}}
    \newcommand{\VariableTok}[1]{\textcolor[rgb]{0.10,0.09,0.49}{{#1}}}
    \newcommand{\ControlFlowTok}[1]{\textcolor[rgb]{0.00,0.44,0.13}{\textbf{{#1}}}}
    \newcommand{\OperatorTok}[1]{\textcolor[rgb]{0.40,0.40,0.40}{{#1}}}
    \newcommand{\BuiltInTok}[1]{{#1}}
    \newcommand{\ExtensionTok}[1]{{#1}}
    \newcommand{\PreprocessorTok}[1]{\textcolor[rgb]{0.74,0.48,0.00}{{#1}}}
    \newcommand{\AttributeTok}[1]{\textcolor[rgb]{0.49,0.56,0.16}{{#1}}}
    \newcommand{\InformationTok}[1]{\textcolor[rgb]{0.38,0.63,0.69}{\textbf{\textit{{#1}}}}}
    \newcommand{\WarningTok}[1]{\textcolor[rgb]{0.38,0.63,0.69}{\textbf{\textit{{#1}}}}}
    
    
    % Define a nice break command that doesn't care if a line doesn't already
    % exist.
    \def\br{\hspace*{\fill} \\* }
    % Math Jax compatibility definitions
    \def\gt{>}
    \def\lt{<}
    \let\Oldtex\TeX
    \let\Oldlatex\LaTeX
    \renewcommand{\TeX}{\textrm{\Oldtex}}
    \renewcommand{\LaTeX}{\textrm{\Oldlatex}}
    % Document parameters
    % Document title
    \title{TP2\_ex1}
    
    
    
    
    
% Pygments definitions
\makeatletter
\def\PY@reset{\let\PY@it=\relax \let\PY@bf=\relax%
    \let\PY@ul=\relax \let\PY@tc=\relax%
    \let\PY@bc=\relax \let\PY@ff=\relax}
\def\PY@tok#1{\csname PY@tok@#1\endcsname}
\def\PY@toks#1+{\ifx\relax#1\empty\else%
    \PY@tok{#1}\expandafter\PY@toks\fi}
\def\PY@do#1{\PY@bc{\PY@tc{\PY@ul{%
    \PY@it{\PY@bf{\PY@ff{#1}}}}}}}
\def\PY#1#2{\PY@reset\PY@toks#1+\relax+\PY@do{#2}}

\@namedef{PY@tok@w}{\def\PY@tc##1{\textcolor[rgb]{0.73,0.73,0.73}{##1}}}
\@namedef{PY@tok@c}{\let\PY@it=\textit\def\PY@tc##1{\textcolor[rgb]{0.24,0.48,0.48}{##1}}}
\@namedef{PY@tok@cp}{\def\PY@tc##1{\textcolor[rgb]{0.61,0.40,0.00}{##1}}}
\@namedef{PY@tok@k}{\let\PY@bf=\textbf\def\PY@tc##1{\textcolor[rgb]{0.00,0.50,0.00}{##1}}}
\@namedef{PY@tok@kp}{\def\PY@tc##1{\textcolor[rgb]{0.00,0.50,0.00}{##1}}}
\@namedef{PY@tok@kt}{\def\PY@tc##1{\textcolor[rgb]{0.69,0.00,0.25}{##1}}}
\@namedef{PY@tok@o}{\def\PY@tc##1{\textcolor[rgb]{0.40,0.40,0.40}{##1}}}
\@namedef{PY@tok@ow}{\let\PY@bf=\textbf\def\PY@tc##1{\textcolor[rgb]{0.67,0.13,1.00}{##1}}}
\@namedef{PY@tok@nb}{\def\PY@tc##1{\textcolor[rgb]{0.00,0.50,0.00}{##1}}}
\@namedef{PY@tok@nf}{\def\PY@tc##1{\textcolor[rgb]{0.00,0.00,1.00}{##1}}}
\@namedef{PY@tok@nc}{\let\PY@bf=\textbf\def\PY@tc##1{\textcolor[rgb]{0.00,0.00,1.00}{##1}}}
\@namedef{PY@tok@nn}{\let\PY@bf=\textbf\def\PY@tc##1{\textcolor[rgb]{0.00,0.00,1.00}{##1}}}
\@namedef{PY@tok@ne}{\let\PY@bf=\textbf\def\PY@tc##1{\textcolor[rgb]{0.80,0.25,0.22}{##1}}}
\@namedef{PY@tok@nv}{\def\PY@tc##1{\textcolor[rgb]{0.10,0.09,0.49}{##1}}}
\@namedef{PY@tok@no}{\def\PY@tc##1{\textcolor[rgb]{0.53,0.00,0.00}{##1}}}
\@namedef{PY@tok@nl}{\def\PY@tc##1{\textcolor[rgb]{0.46,0.46,0.00}{##1}}}
\@namedef{PY@tok@ni}{\let\PY@bf=\textbf\def\PY@tc##1{\textcolor[rgb]{0.44,0.44,0.44}{##1}}}
\@namedef{PY@tok@na}{\def\PY@tc##1{\textcolor[rgb]{0.41,0.47,0.13}{##1}}}
\@namedef{PY@tok@nt}{\let\PY@bf=\textbf\def\PY@tc##1{\textcolor[rgb]{0.00,0.50,0.00}{##1}}}
\@namedef{PY@tok@nd}{\def\PY@tc##1{\textcolor[rgb]{0.67,0.13,1.00}{##1}}}
\@namedef{PY@tok@s}{\def\PY@tc##1{\textcolor[rgb]{0.73,0.13,0.13}{##1}}}
\@namedef{PY@tok@sd}{\let\PY@it=\textit\def\PY@tc##1{\textcolor[rgb]{0.73,0.13,0.13}{##1}}}
\@namedef{PY@tok@si}{\let\PY@bf=\textbf\def\PY@tc##1{\textcolor[rgb]{0.64,0.35,0.47}{##1}}}
\@namedef{PY@tok@se}{\let\PY@bf=\textbf\def\PY@tc##1{\textcolor[rgb]{0.67,0.36,0.12}{##1}}}
\@namedef{PY@tok@sr}{\def\PY@tc##1{\textcolor[rgb]{0.64,0.35,0.47}{##1}}}
\@namedef{PY@tok@ss}{\def\PY@tc##1{\textcolor[rgb]{0.10,0.09,0.49}{##1}}}
\@namedef{PY@tok@sx}{\def\PY@tc##1{\textcolor[rgb]{0.00,0.50,0.00}{##1}}}
\@namedef{PY@tok@m}{\def\PY@tc##1{\textcolor[rgb]{0.40,0.40,0.40}{##1}}}
\@namedef{PY@tok@gh}{\let\PY@bf=\textbf\def\PY@tc##1{\textcolor[rgb]{0.00,0.00,0.50}{##1}}}
\@namedef{PY@tok@gu}{\let\PY@bf=\textbf\def\PY@tc##1{\textcolor[rgb]{0.50,0.00,0.50}{##1}}}
\@namedef{PY@tok@gd}{\def\PY@tc##1{\textcolor[rgb]{0.63,0.00,0.00}{##1}}}
\@namedef{PY@tok@gi}{\def\PY@tc##1{\textcolor[rgb]{0.00,0.52,0.00}{##1}}}
\@namedef{PY@tok@gr}{\def\PY@tc##1{\textcolor[rgb]{0.89,0.00,0.00}{##1}}}
\@namedef{PY@tok@ge}{\let\PY@it=\textit}
\@namedef{PY@tok@gs}{\let\PY@bf=\textbf}
\@namedef{PY@tok@gp}{\let\PY@bf=\textbf\def\PY@tc##1{\textcolor[rgb]{0.00,0.00,0.50}{##1}}}
\@namedef{PY@tok@go}{\def\PY@tc##1{\textcolor[rgb]{0.44,0.44,0.44}{##1}}}
\@namedef{PY@tok@gt}{\def\PY@tc##1{\textcolor[rgb]{0.00,0.27,0.87}{##1}}}
\@namedef{PY@tok@err}{\def\PY@bc##1{{\setlength{\fboxsep}{\string -\fboxrule}\fcolorbox[rgb]{1.00,0.00,0.00}{1,1,1}{\strut ##1}}}}
\@namedef{PY@tok@kc}{\let\PY@bf=\textbf\def\PY@tc##1{\textcolor[rgb]{0.00,0.50,0.00}{##1}}}
\@namedef{PY@tok@kd}{\let\PY@bf=\textbf\def\PY@tc##1{\textcolor[rgb]{0.00,0.50,0.00}{##1}}}
\@namedef{PY@tok@kn}{\let\PY@bf=\textbf\def\PY@tc##1{\textcolor[rgb]{0.00,0.50,0.00}{##1}}}
\@namedef{PY@tok@kr}{\let\PY@bf=\textbf\def\PY@tc##1{\textcolor[rgb]{0.00,0.50,0.00}{##1}}}
\@namedef{PY@tok@bp}{\def\PY@tc##1{\textcolor[rgb]{0.00,0.50,0.00}{##1}}}
\@namedef{PY@tok@fm}{\def\PY@tc##1{\textcolor[rgb]{0.00,0.00,1.00}{##1}}}
\@namedef{PY@tok@vc}{\def\PY@tc##1{\textcolor[rgb]{0.10,0.09,0.49}{##1}}}
\@namedef{PY@tok@vg}{\def\PY@tc##1{\textcolor[rgb]{0.10,0.09,0.49}{##1}}}
\@namedef{PY@tok@vi}{\def\PY@tc##1{\textcolor[rgb]{0.10,0.09,0.49}{##1}}}
\@namedef{PY@tok@vm}{\def\PY@tc##1{\textcolor[rgb]{0.10,0.09,0.49}{##1}}}
\@namedef{PY@tok@sa}{\def\PY@tc##1{\textcolor[rgb]{0.73,0.13,0.13}{##1}}}
\@namedef{PY@tok@sb}{\def\PY@tc##1{\textcolor[rgb]{0.73,0.13,0.13}{##1}}}
\@namedef{PY@tok@sc}{\def\PY@tc##1{\textcolor[rgb]{0.73,0.13,0.13}{##1}}}
\@namedef{PY@tok@dl}{\def\PY@tc##1{\textcolor[rgb]{0.73,0.13,0.13}{##1}}}
\@namedef{PY@tok@s2}{\def\PY@tc##1{\textcolor[rgb]{0.73,0.13,0.13}{##1}}}
\@namedef{PY@tok@sh}{\def\PY@tc##1{\textcolor[rgb]{0.73,0.13,0.13}{##1}}}
\@namedef{PY@tok@s1}{\def\PY@tc##1{\textcolor[rgb]{0.73,0.13,0.13}{##1}}}
\@namedef{PY@tok@mb}{\def\PY@tc##1{\textcolor[rgb]{0.40,0.40,0.40}{##1}}}
\@namedef{PY@tok@mf}{\def\PY@tc##1{\textcolor[rgb]{0.40,0.40,0.40}{##1}}}
\@namedef{PY@tok@mh}{\def\PY@tc##1{\textcolor[rgb]{0.40,0.40,0.40}{##1}}}
\@namedef{PY@tok@mi}{\def\PY@tc##1{\textcolor[rgb]{0.40,0.40,0.40}{##1}}}
\@namedef{PY@tok@il}{\def\PY@tc##1{\textcolor[rgb]{0.40,0.40,0.40}{##1}}}
\@namedef{PY@tok@mo}{\def\PY@tc##1{\textcolor[rgb]{0.40,0.40,0.40}{##1}}}
\@namedef{PY@tok@ch}{\let\PY@it=\textit\def\PY@tc##1{\textcolor[rgb]{0.24,0.48,0.48}{##1}}}
\@namedef{PY@tok@cm}{\let\PY@it=\textit\def\PY@tc##1{\textcolor[rgb]{0.24,0.48,0.48}{##1}}}
\@namedef{PY@tok@cpf}{\let\PY@it=\textit\def\PY@tc##1{\textcolor[rgb]{0.24,0.48,0.48}{##1}}}
\@namedef{PY@tok@c1}{\let\PY@it=\textit\def\PY@tc##1{\textcolor[rgb]{0.24,0.48,0.48}{##1}}}
\@namedef{PY@tok@cs}{\let\PY@it=\textit\def\PY@tc##1{\textcolor[rgb]{0.24,0.48,0.48}{##1}}}

\def\PYZbs{\char`\\}
\def\PYZus{\char`\_}
\def\PYZob{\char`\{}
\def\PYZcb{\char`\}}
\def\PYZca{\char`\^}
\def\PYZam{\char`\&}
\def\PYZlt{\char`\<}
\def\PYZgt{\char`\>}
\def\PYZsh{\char`\#}
\def\PYZpc{\char`\%}
\def\PYZdl{\char`\$}
\def\PYZhy{\char`\-}
\def\PYZsq{\char`\'}
\def\PYZdq{\char`\"}
\def\PYZti{\char`\~}
% for compatibility with earlier versions
\def\PYZat{@}
\def\PYZlb{[}
\def\PYZrb{]}
\makeatother


    % For linebreaks inside Verbatim environment from package fancyvrb. 
    \makeatletter
        \newbox\Wrappedcontinuationbox 
        \newbox\Wrappedvisiblespacebox 
        \newcommand*\Wrappedvisiblespace {\textcolor{red}{\textvisiblespace}} 
        \newcommand*\Wrappedcontinuationsymbol {\textcolor{red}{\llap{\tiny$\m@th\hookrightarrow$}}} 
        \newcommand*\Wrappedcontinuationindent {3ex } 
        \newcommand*\Wrappedafterbreak {\kern\Wrappedcontinuationindent\copy\Wrappedcontinuationbox} 
        % Take advantage of the already applied Pygments mark-up to insert 
        % potential linebreaks for TeX processing. 
        %        {, <, #, %, $, ' and ": go to next line. 
        %        _, }, ^, &, >, - and ~: stay at end of broken line. 
        % Use of \textquotesingle for straight quote. 
        \newcommand*\Wrappedbreaksatspecials {% 
            \def\PYGZus{\discretionary{\char`\_}{\Wrappedafterbreak}{\char`\_}}% 
            \def\PYGZob{\discretionary{}{\Wrappedafterbreak\char`\{}{\char`\{}}% 
            \def\PYGZcb{\discretionary{\char`\}}{\Wrappedafterbreak}{\char`\}}}% 
            \def\PYGZca{\discretionary{\char`\^}{\Wrappedafterbreak}{\char`\^}}% 
            \def\PYGZam{\discretionary{\char`\&}{\Wrappedafterbreak}{\char`\&}}% 
            \def\PYGZlt{\discretionary{}{\Wrappedafterbreak\char`\<}{\char`\<}}% 
            \def\PYGZgt{\discretionary{\char`\>}{\Wrappedafterbreak}{\char`\>}}% 
            \def\PYGZsh{\discretionary{}{\Wrappedafterbreak\char`\#}{\char`\#}}% 
            \def\PYGZpc{\discretionary{}{\Wrappedafterbreak\char`\%}{\char`\%}}% 
            \def\PYGZdl{\discretionary{}{\Wrappedafterbreak\char`\$}{\char`\$}}% 
            \def\PYGZhy{\discretionary{\char`\-}{\Wrappedafterbreak}{\char`\-}}% 
            \def\PYGZsq{\discretionary{}{\Wrappedafterbreak\textquotesingle}{\textquotesingle}}% 
            \def\PYGZdq{\discretionary{}{\Wrappedafterbreak\char`\"}{\char`\"}}% 
            \def\PYGZti{\discretionary{\char`\~}{\Wrappedafterbreak}{\char`\~}}% 
        } 
        % Some characters . , ; ? ! / are not pygmentized. 
        % This macro makes them "active" and they will insert potential linebreaks 
        \newcommand*\Wrappedbreaksatpunct {% 
            \lccode`\~`\.\lowercase{\def~}{\discretionary{\hbox{\char`\.}}{\Wrappedafterbreak}{\hbox{\char`\.}}}% 
            \lccode`\~`\,\lowercase{\def~}{\discretionary{\hbox{\char`\,}}{\Wrappedafterbreak}{\hbox{\char`\,}}}% 
            \lccode`\~`\;\lowercase{\def~}{\discretionary{\hbox{\char`\;}}{\Wrappedafterbreak}{\hbox{\char`\;}}}% 
            \lccode`\~`\:\lowercase{\def~}{\discretionary{\hbox{\char`\:}}{\Wrappedafterbreak}{\hbox{\char`\:}}}% 
            \lccode`\~`\?\lowercase{\def~}{\discretionary{\hbox{\char`\?}}{\Wrappedafterbreak}{\hbox{\char`\?}}}% 
            \lccode`\~`\!\lowercase{\def~}{\discretionary{\hbox{\char`\!}}{\Wrappedafterbreak}{\hbox{\char`\!}}}% 
            \lccode`\~`\/\lowercase{\def~}{\discretionary{\hbox{\char`\/}}{\Wrappedafterbreak}{\hbox{\char`\/}}}% 
            \catcode`\.\active
            \catcode`\,\active 
            \catcode`\;\active
            \catcode`\:\active
            \catcode`\?\active
            \catcode`\!\active
            \catcode`\/\active 
            \lccode`\~`\~ 	
        }
    \makeatother

    \let\OriginalVerbatim=\Verbatim
    \makeatletter
    \renewcommand{\Verbatim}[1][1]{%
        %\parskip\z@skip
        \sbox\Wrappedcontinuationbox {\Wrappedcontinuationsymbol}%
        \sbox\Wrappedvisiblespacebox {\FV@SetupFont\Wrappedvisiblespace}%
        \def\FancyVerbFormatLine ##1{\hsize\linewidth
            \vtop{\raggedright\hyphenpenalty\z@\exhyphenpenalty\z@
                \doublehyphendemerits\z@\finalhyphendemerits\z@
                \strut ##1\strut}%
        }%
        % If the linebreak is at a space, the latter will be displayed as visible
        % space at end of first line, and a continuation symbol starts next line.
        % Stretch/shrink are however usually zero for typewriter font.
        \def\FV@Space {%
            \nobreak\hskip\z@ plus\fontdimen3\font minus\fontdimen4\font
            \discretionary{\copy\Wrappedvisiblespacebox}{\Wrappedafterbreak}
            {\kern\fontdimen2\font}%
        }%
        
        % Allow breaks at special characters using \PYG... macros.
        \Wrappedbreaksatspecials
        % Breaks at punctuation characters . , ; ? ! and / need catcode=\active 	
        \OriginalVerbatim[#1,codes*=\Wrappedbreaksatpunct]%
    }
    \makeatother

    % Exact colors from NB
    \definecolor{incolor}{HTML}{303F9F}
    \definecolor{outcolor}{HTML}{D84315}
    \definecolor{cellborder}{HTML}{CFCFCF}
    \definecolor{cellbackground}{HTML}{F7F7F7}
    
    % prompt
    \makeatletter
    \newcommand{\boxspacing}{\kern\kvtcb@left@rule\kern\kvtcb@boxsep}
    \makeatother
    \newcommand{\prompt}[4]{
        {\ttfamily\llap{{\color{#2}[#3]:\hspace{3pt}#4}}\vspace{-\baselineskip}}
    }
    

    
    % Prevent overflowing lines due to hard-to-break entities
    \sloppy 
    % Setup hyperref package
    \hypersetup{
      breaklinks=true,  % so long urls are correctly broken across lines
      colorlinks=true,
      urlcolor=urlcolor,
      linkcolor=linkcolor,
      citecolor=citecolor,
      }
    % Slightly bigger margins than the latex defaults
    
    \geometry{verbose,tmargin=1in,bmargin=1in,lmargin=1in,rmargin=1in}
    
    

\begin{document}
    
    \maketitle
    
    

    
    \hypertarget{implementauxe7uxe3o}{%
\textbf{\large Implementação}\label{implementauxe7uxe3o}}

Começamos por importar o módulo pysmt.shortcuts que oferece uma API
simplificada que disponibiliza as funcionalidades para a utilização
usual de um SMT solver. Os tipos estão definidos no módulo pysmt.typing
de onde temos que importar o tipo INT e o BVType (para utilização de bit
vectors)

    \begin{tcolorbox}[breakable, size=fbox, boxrule=1pt, pad at break*=1mm,colback=cellbackground, colframe=cellborder]
\prompt{In}{incolor}{1}{\boxspacing}
\begin{Verbatim}[commandchars=\\\{\}]
\PY{k+kn}{from} \PY{n+nn}{pysmt}\PY{n+nn}{.}\PY{n+nn}{shortcuts} \PY{k+kn}{import} \PY{o}{*}
\PY{k+kn}{from} \PY{n+nn}{pysmt}\PY{n+nn}{.}\PY{n+nn}{typing} \PY{k+kn}{import} \PY{n}{INT}
\end{Verbatim}
\end{tcolorbox}

    A seguinte função cria a 𝑖-ésima cópia das variáveis de estado,
agrupadas num dicionário que nos permite aceder às mesmas pelo nome.

    \hypertarget{paruxe2metros-do-programa}{%
    \textbf{\large Parâmetros do programa}\label{paruxe2metros-do-programa}}

\begin{itemize}
\tightlist
\item
  \textbf{a} e \textbf{b}: dois valores inteiros que serão multiplicados
\item
  \textbf{n}: precisão limitada (em bits)
\end{itemize}

    \begin{tcolorbox}[breakable, size=fbox, boxrule=1pt, pad at break*=1mm,colback=cellbackground, colframe=cellborder]
\prompt{In}{incolor}{2}{\boxspacing}
\begin{Verbatim}[commandchars=\\\{\}]
\PY{n}{a} \PY{o}{=} \PY{l+m+mi}{4}
\PY{n}{b} \PY{o}{=} \PY{l+m+mi}{2}
\PY{n}{n} \PY{o}{=} \PY{l+m+mi}{4} 
\end{Verbatim}
\end{tcolorbox}

    \hypertarget{funuxe7uxe3o-declare}{%
    \textbf{\large Função declare}\label{funuxe7uxe3o-declare}}

A seguinte função cria a 𝑖-ésima cópia das variáveis de estado,
agrupadas num dicionário que nos permite aceder às mesmas pelo nome.

\textbf{x}, \textbf{y} e \textbf{z} são bit vectors e \textbf{pc}
(program counter) é um inteiro

    \begin{tcolorbox}[breakable, size=fbox, boxrule=1pt, pad at break*=1mm,colback=cellbackground, colframe=cellborder]
\prompt{In}{incolor}{3}{\boxspacing}
\begin{Verbatim}[commandchars=\\\{\}]
\PY{k}{def} \PY{n+nf}{declare}\PY{p}{(}\PY{n}{i}\PY{p}{)}\PY{p}{:}
    \PY{n}{state} \PY{o}{=} \PY{p}{\PYZob{}}\PY{p}{\PYZcb{}}
    \PY{n}{state}\PY{p}{[}\PY{l+s+s1}{\PYZsq{}}\PY{l+s+s1}{pc}\PY{l+s+s1}{\PYZsq{}}\PY{p}{]} \PY{o}{=} \PY{n}{Symbol}\PY{p}{(}\PY{l+s+s1}{\PYZsq{}}\PY{l+s+s1}{pc}\PY{l+s+s1}{\PYZsq{}} \PY{o}{+} \PY{n+nb}{str}\PY{p}{(}\PY{n}{i}\PY{p}{)}\PY{p}{,} \PY{n}{INT}\PY{p}{)}
    \PY{n}{state}\PY{p}{[}\PY{l+s+s1}{\PYZsq{}}\PY{l+s+s1}{x}\PY{l+s+s1}{\PYZsq{}}\PY{p}{]} \PY{o}{=} \PY{n}{Symbol}\PY{p}{(}\PY{l+s+s1}{\PYZsq{}}\PY{l+s+s1}{x}\PY{l+s+s1}{\PYZsq{}} \PY{o}{+} \PY{n+nb}{str}\PY{p}{(}\PY{n}{i}\PY{p}{)}\PY{p}{,} \PY{n}{BVType}\PY{p}{(}\PY{n}{n}\PY{p}{)}\PY{p}{)}
    \PY{n}{state}\PY{p}{[}\PY{l+s+s1}{\PYZsq{}}\PY{l+s+s1}{y}\PY{l+s+s1}{\PYZsq{}}\PY{p}{]} \PY{o}{=} \PY{n}{Symbol}\PY{p}{(}\PY{l+s+s1}{\PYZsq{}}\PY{l+s+s1}{y}\PY{l+s+s1}{\PYZsq{}} \PY{o}{+} \PY{n+nb}{str}\PY{p}{(}\PY{n}{i}\PY{p}{)}\PY{p}{,} \PY{n}{BVType}\PY{p}{(}\PY{n}{n}\PY{p}{)}\PY{p}{)}
    \PY{n}{state}\PY{p}{[}\PY{l+s+s1}{\PYZsq{}}\PY{l+s+s1}{z}\PY{l+s+s1}{\PYZsq{}}\PY{p}{]} \PY{o}{=} \PY{n}{Symbol}\PY{p}{(}\PY{l+s+s1}{\PYZsq{}}\PY{l+s+s1}{z}\PY{l+s+s1}{\PYZsq{}} \PY{o}{+} \PY{n+nb}{str}\PY{p}{(}\PY{n}{i}\PY{p}{)}\PY{p}{,} \PY{n}{BVType}\PY{p}{(}\PY{n}{n}\PY{p}{)}\PY{p}{)}
    \PY{k}{return} \PY{n}{state}
\end{Verbatim}
\end{tcolorbox}

    \hypertarget{funuxe7uxe3o-init}{%
    \textbf{\large Função init}\label{funuxe7uxe3o-init}}

Dado um possível estado do programa (um dicionário de variáveis),
devolve um predicado do pySMT que testa se esse estado é um possível
estado inicial do programa.

Analisando o Control Flow Automaton do enunciado do problema, definimos
que o estado inicial corresponde a esta parte do diagrama:

\begin{figure}
\centering
\includegraphics{init.png}
\caption{init.png}
\end{figure}

O diagrama exige que neste estado tenhamos:

\begin{itemize}
\tightlist
\item
  x igual a \textbf{a} (a é parâmetro do programa e é inteiro)
\item
  y igual a \textbf{b} (b é parâmetro do programa e é inteiro)
\item
  z igual a \textbf{0}
\end{itemize}

Adicionalmente, definimos que o \textbf{program counter} do nosso estado
inicial é 0

Como \textbf{x}, \textbf{y}, \textbf{z} serão bit vectors, construímos
um predicado assumindo os valores inteiros enunciados acima no formato
de bit vectors com precisão limitada de \textbf{n} bits

    \begin{tcolorbox}[breakable, size=fbox, boxrule=1pt, pad at break*=1mm,colback=cellbackground, colframe=cellborder]
\prompt{In}{incolor}{4}{\boxspacing}
\begin{Verbatim}[commandchars=\\\{\}]
\PY{k}{def} \PY{n+nf}{init}\PY{p}{(}\PY{n}{state}\PY{p}{)}\PY{p}{:}
    \PY{k}{return} \PY{n}{And}\PY{p}{(}\PY{n}{Equals}\PY{p}{(}\PY{n}{state}\PY{p}{[}\PY{l+s+s1}{\PYZsq{}}\PY{l+s+s1}{pc}\PY{l+s+s1}{\PYZsq{}}\PY{p}{]}\PY{p}{,} \PY{n}{Int}\PY{p}{(}\PY{l+m+mi}{0}\PY{p}{)}\PY{p}{)}\PY{p}{,} \PY{n}{Equals}\PY{p}{(}\PY{n}{state}\PY{p}{[}\PY{l+s+s1}{\PYZsq{}}\PY{l+s+s1}{x}\PY{l+s+s1}{\PYZsq{}}\PY{p}{]}\PY{p}{,} \PY{n}{BV}\PY{p}{(}\PY{n}{a}\PY{p}{,} \PY{n}{n}\PY{p}{)}\PY{p}{)}\PY{p}{,} \PY{n}{Equals}\PY{p}{(}\PY{n}{state}\PY{p}{[}\PY{l+s+s1}{\PYZsq{}}\PY{l+s+s1}{y}\PY{l+s+s1}{\PYZsq{}}\PY{p}{]}\PY{p}{,} \PY{n}{BV}\PY{p}{(}\PY{n}{b}\PY{p}{,} \PY{n}{n}\PY{p}{)}\PY{p}{)}\PY{p}{,} \PY{n}{Equals}\PY{p}{(}\PY{n}{state}\PY{p}{[}\PY{l+s+s1}{\PYZsq{}}\PY{l+s+s1}{z}\PY{l+s+s1}{\PYZsq{}}\PY{p}{]}\PY{p}{,} \PY{n}{BVZero}\PY{p}{(}\PY{n}{n}\PY{p}{)}\PY{p}{)}\PY{p}{)}
\end{Verbatim}
\end{tcolorbox}

    \hypertarget{funuxe7uxe3o-trans}{%
    \textbf{\large Função trans}\label{funuxe7uxe3o-trans}}

Dados dois possíveis estados do programa, devolve um predicado do pySMT
que testa se é possível transitar do primeiro para o segundo.

Esta função contém todas as transições possíveis dentro do programa,
basta que uma se verifique para que possamos transitar do estado
\textbf{curr} para o \textbf{prox}

Vejamos o que cada transição representa no Control Flow Automaton
ilustrado no enunciado do trabalho

\hypertarget{t1}{%
\textbf{\Large t1}\label{t1}}

\begin{itemize}
\tightlist
\item
  Ocorre imediatamente após o estado inicial
\item
  Representa a passagem do \textbf{program counter} (\textbf{pc}) de 0
  para 1
\item
  Os valores de \textbf{x}, \textbf{y}, \textbf{z} mantém-se iguais
\end{itemize}

\begin{figure}
\centering
\includegraphics{image.png}
\caption{image.png}
\end{figure}

\hypertarget{t2-nuxe3o-overflow-e-t3-overflow}{%
\textbf{\Large t2 (não overflow) e t3
(overflow)}\label{t2-nuxe3o-overflow-e-t3-overflow}}

t2:

\begin{itemize}
\tightlist
\item
  Ocorre quando o valor de y é par, isto é, quando o bit menos
  significativo é 0 e quando y é diferente de 0 e quando não há overflow
\item
  Representa a passagem do \textbf{program counter} (\textbf{pc}) de 1
  para 2
\item
  O valor de \textbf{x} é duplicado
\item
  O valor de \textbf{y} é dividido por 2
\item
  O valor de \textbf{z} mantém-se
\end{itemize}

t3:

\begin{itemize}
\tightlist
\item
  Ocorre quando o valor de y é par, isto é, quando o bit menos
  significativo é 0 e quando y é diferente de 0 e quando há overflow
\item
  Representa a passagem do \textbf{program counter} (\textbf{pc}) de 1
  para -1
\item
  O valor de \textbf{x} é duplicado
\item
  O valor de \textbf{y} é dividido por 2
\item
  O valor de \textbf{z} mantém-se
\end{itemize}

\begin{figure}
\centering
\includegraphics{image-2.png}
\caption{image-2.png}
\end{figure}

\hypertarget{t4}{%
\textbf{\Large t4}\label{t4}}

\begin{itemize}
\tightlist
\item
  Ocorre após imediatamente após efetuar t2
\item
  Representa a passagem do \textbf{program counter} (\textbf{pc}) de 2
  para 1
\item
  Os valores de \textbf{x}, \textbf{y}, \textbf{z} mantém-se iguais
\end{itemize}

\begin{figure}
\centering
\includegraphics{image-3.png}
\caption{image-3.png}
\end{figure}
\pagebreak
\hypertarget{t5-nuxe3o-overflow-e-t6-overflow}{%
\textbf{\Large t5 (não overflow) e t6
(overflow)}\label{t5-nuxe3o-overflow-e-t6-overflow}}

t5:

\begin{itemize}
\tightlist
\item
  Ocorre quando o valor de y é ímpar, isto é, quando o bit menos
  significativo é 1 e quando y é diferente de 0 e quando não há overflow
\item
  Representa a passagem do \textbf{program counter} (\textbf{pc}) de 1
  para 3
\item
  O valor de \textbf{x} mantém-se
\item
  Ao valor de \textbf{y} subtraí-se 1
\item
  Ao valor de \textbf{z} adiciona-se o valor de \textbf{x}
\end{itemize}

t6:

\begin{itemize}
\tightlist
\item
  Ocorre quando o valor de y é ímpar, isto é, quando o bit menos
  significativo é 1 e quando y é diferente de 0 e quando há overflow
\item
  Representa a passagem do \textbf{program counter} (\textbf{pc}) de 1
  para -1
\item
  O valor de \textbf{x} mantém-se
\item
  Ao valor de \textbf{y} subtraí-se 1
\item
  Ao valor de \textbf{z} adiciona-se o valor de \textbf{x}
\end{itemize}

\begin{figure}
\centering
\includegraphics{image-4.png}
\caption{image-4.png}
\end{figure}
\pagebreak
\hypertarget{t7}{%
\textbf{\Large t7}\label{t7}}

\begin{itemize}
\tightlist
\item
  Ocorre após imediatamente após efetuar t5
\item
  Representa a passagem do \textbf{program counter} (\textbf{pc}) de 3
  para 1
\item
  Os valores de \textbf{x}, \textbf{y}, \textbf{z} mantém-se iguais
\end{itemize}

\begin{figure}
\centering
\includegraphics{image-5.png}
\caption{image-5.png}
\end{figure}

\hypertarget{t8}{%
\textbf{\Large t8}\label{t8}}

\begin{itemize}
\tightlist
\item
  Ocorre quando o valor de y é 0
\item
  Representa a passagem do \textbf{program counter} (\textbf{pc}) de 1
  para 4
\item
  Os valores de \textbf{x}, \textbf{y}, \textbf{z} mantém-se iguais
\end{itemize}

\begin{figure}
\centering
\includegraphics{image-6.png}
\caption{image-6.png}
\end{figure}

\hypertarget{tstop}{%
\textbf{\Large tSTOP}\label{tstop}}

\begin{itemize}
\tightlist
\item
  Ocorre quando o programa acabou com sucesso
\item
  Representa a passagem do \textbf{program counter} (\textbf{pc}) de 4
  para 4
\end{itemize}

\begin{figure}
\centering
\includegraphics{image-7.png}
\caption{image-7.png}
\end{figure}

\hypertarget{terror}{%
\textbf{\Large tERROR}\label{terror}}

\begin{itemize}
\tightlist
\item
  Ocorre quando o programa acabou sem sucesso
\item
  Representa a passagem do \textbf{program counter} (\textbf{pc}) de -1
  para -1
\end{itemize}

\begin{figure}
\centering
\includegraphics{image-8.png}
\caption{image-8.png}
\end{figure}

    \begin{tcolorbox}[breakable, size=fbox, boxrule=1pt, pad at break*=1mm,colback=cellbackground, colframe=cellborder]
\prompt{In}{incolor}{5}{\boxspacing}
\begin{Verbatim}[commandchars=\\\{\}]
\PY{k}{def} \PY{n+nf}{trans}\PY{p}{(}\PY{n}{curr}\PY{p}{,}\PY{n}{prox}\PY{p}{)}\PY{p}{:}
    \PY{n}{t1} \PY{o}{=} \PY{n}{And}\PY{p}{(}
            \PY{n}{Equals}\PY{p}{(}\PY{n}{curr}\PY{p}{[}\PY{l+s+s1}{\PYZsq{}}\PY{l+s+s1}{pc}\PY{l+s+s1}{\PYZsq{}}\PY{p}{]}\PY{p}{,} \PY{n}{Int}\PY{p}{(}\PY{l+m+mi}{0}\PY{p}{)}\PY{p}{)}\PY{p}{,}
            \PY{n}{Equals}\PY{p}{(}\PY{n}{prox}\PY{p}{[}\PY{l+s+s1}{\PYZsq{}}\PY{l+s+s1}{pc}\PY{l+s+s1}{\PYZsq{}}\PY{p}{]}\PY{p}{,} \PY{n}{Int}\PY{p}{(}\PY{l+m+mi}{1}\PY{p}{)}\PY{p}{)}\PY{p}{,}
            \PY{n}{Equals}\PY{p}{(}\PY{n}{prox}\PY{p}{[}\PY{l+s+s1}{\PYZsq{}}\PY{l+s+s1}{x}\PY{l+s+s1}{\PYZsq{}}\PY{p}{]}\PY{p}{,} \PY{n}{curr}\PY{p}{[}\PY{l+s+s1}{\PYZsq{}}\PY{l+s+s1}{x}\PY{l+s+s1}{\PYZsq{}}\PY{p}{]}\PY{p}{)}\PY{p}{,} 
            \PY{n}{Equals}\PY{p}{(}\PY{n}{prox}\PY{p}{[}\PY{l+s+s1}{\PYZsq{}}\PY{l+s+s1}{y}\PY{l+s+s1}{\PYZsq{}}\PY{p}{]}\PY{p}{,} \PY{n}{curr}\PY{p}{[}\PY{l+s+s1}{\PYZsq{}}\PY{l+s+s1}{y}\PY{l+s+s1}{\PYZsq{}}\PY{p}{]}\PY{p}{)}\PY{p}{,} 
            \PY{n}{Equals}\PY{p}{(}\PY{n}{prox}\PY{p}{[}\PY{l+s+s1}{\PYZsq{}}\PY{l+s+s1}{z}\PY{l+s+s1}{\PYZsq{}}\PY{p}{]}\PY{p}{,} \PY{n}{curr}\PY{p}{[}\PY{l+s+s1}{\PYZsq{}}\PY{l+s+s1}{z}\PY{l+s+s1}{\PYZsq{}}\PY{p}{]}\PY{p}{)} 
        \PY{p}{)}     

    \PY{n}{t2} \PY{o}{=} \PY{n}{And}\PY{p}{(}
            \PY{n}{Equals}\PY{p}{(}\PY{n}{curr}\PY{p}{[}\PY{l+s+s1}{\PYZsq{}}\PY{l+s+s1}{pc}\PY{l+s+s1}{\PYZsq{}}\PY{p}{]}\PY{p}{,} \PY{n}{Int}\PY{p}{(}\PY{l+m+mi}{1}\PY{p}{)}\PY{p}{)}\PY{p}{,}
            \PY{n}{Equals}\PY{p}{(}\PY{n}{prox}\PY{p}{[}\PY{l+s+s1}{\PYZsq{}}\PY{l+s+s1}{pc}\PY{l+s+s1}{\PYZsq{}}\PY{p}{]}\PY{p}{,} \PY{n}{Int}\PY{p}{(}\PY{l+m+mi}{2}\PY{p}{)}\PY{p}{)}\PY{p}{,} 
            \PY{n}{NotEquals}\PY{p}{(}\PY{n}{curr}\PY{p}{[}\PY{l+s+s1}{\PYZsq{}}\PY{l+s+s1}{y}\PY{l+s+s1}{\PYZsq{}}\PY{p}{]}\PY{p}{,} \PY{n}{BVZero}\PY{p}{(}\PY{n}{n}\PY{p}{)}\PY{p}{)}\PY{p}{,} 
            \PY{n}{Equals}\PY{p}{(}\PY{n}{BVZero}\PY{p}{(}\PY{l+m+mi}{1}\PY{p}{)}\PY{p}{,} \PY{n}{BVExtract}\PY{p}{(}\PY{n}{curr}\PY{p}{[}\PY{l+s+s1}{\PYZsq{}}\PY{l+s+s1}{y}\PY{l+s+s1}{\PYZsq{}}\PY{p}{]}\PY{p}{,} \PY{n}{start}\PY{o}{=}\PY{l+m+mi}{0}\PY{p}{,} \PY{n}{end}\PY{o}{=}\PY{l+m+mi}{0}\PY{p}{)}\PY{p}{)}\PY{p}{,}
            \PY{n}{Equals}\PY{p}{(}\PY{n}{prox}\PY{p}{[}\PY{l+s+s1}{\PYZsq{}}\PY{l+s+s1}{x}\PY{l+s+s1}{\PYZsq{}}\PY{p}{]}\PY{p}{,} \PY{n}{BVMul}\PY{p}{(}\PY{n}{curr}\PY{p}{[}\PY{l+s+s1}{\PYZsq{}}\PY{l+s+s1}{x}\PY{l+s+s1}{\PYZsq{}}\PY{p}{]}\PY{p}{,} \PY{n}{BV}\PY{p}{(}\PY{l+m+mi}{2}\PY{p}{,} \PY{n}{n}\PY{p}{)}\PY{p}{)}\PY{p}{)}\PY{p}{,}
            \PY{n}{Equals}\PY{p}{(}\PY{n}{prox}\PY{p}{[}\PY{l+s+s1}{\PYZsq{}}\PY{l+s+s1}{y}\PY{l+s+s1}{\PYZsq{}}\PY{p}{]}\PY{p}{,} \PY{n}{BVUDiv}\PY{p}{(}\PY{n}{curr}\PY{p}{[}\PY{l+s+s1}{\PYZsq{}}\PY{l+s+s1}{y}\PY{l+s+s1}{\PYZsq{}}\PY{p}{]}\PY{p}{,} \PY{n}{BV}\PY{p}{(}\PY{l+m+mi}{2}\PY{p}{,} \PY{n}{n}\PY{p}{)}\PY{p}{)}\PY{p}{)}\PY{p}{,}
            \PY{n}{Equals}\PY{p}{(}\PY{n}{prox}\PY{p}{[}\PY{l+s+s1}{\PYZsq{}}\PY{l+s+s1}{z}\PY{l+s+s1}{\PYZsq{}}\PY{p}{]}\PY{p}{,} \PY{n}{curr}\PY{p}{[}\PY{l+s+s1}{\PYZsq{}}\PY{l+s+s1}{z}\PY{l+s+s1}{\PYZsq{}}\PY{p}{]}\PY{p}{)}\PY{p}{,}
            \PY{n}{Equals}\PY{p}{(}\PY{n}{BVUDiv}\PY{p}{(}\PY{n}{prox}\PY{p}{[}\PY{l+s+s1}{\PYZsq{}}\PY{l+s+s1}{x}\PY{l+s+s1}{\PYZsq{}}\PY{p}{]}\PY{p}{,} \PY{n}{curr}\PY{p}{[}\PY{l+s+s1}{\PYZsq{}}\PY{l+s+s1}{x}\PY{l+s+s1}{\PYZsq{}}\PY{p}{]}\PY{p}{)}\PY{p}{,} \PY{n}{BV}\PY{p}{(}\PY{l+m+mi}{2}\PY{p}{,} \PY{n}{n}\PY{p}{)}\PY{p}{)}\PY{p}{,}
            \PY{p}{)}

    \PY{n}{t3} \PY{o}{=} \PY{n}{And}\PY{p}{(}
            \PY{n}{Equals}\PY{p}{(}\PY{n}{curr}\PY{p}{[}\PY{l+s+s1}{\PYZsq{}}\PY{l+s+s1}{pc}\PY{l+s+s1}{\PYZsq{}}\PY{p}{]}\PY{p}{,} \PY{n}{Int}\PY{p}{(}\PY{l+m+mi}{1}\PY{p}{)}\PY{p}{)}\PY{p}{,}
            \PY{n}{Equals}\PY{p}{(}\PY{n}{prox}\PY{p}{[}\PY{l+s+s1}{\PYZsq{}}\PY{l+s+s1}{pc}\PY{l+s+s1}{\PYZsq{}}\PY{p}{]}\PY{p}{,} \PY{n}{Int}\PY{p}{(}\PY{o}{\PYZhy{}}\PY{l+m+mi}{1}\PY{p}{)}\PY{p}{)}\PY{p}{,} 
            \PY{n}{NotEquals}\PY{p}{(}\PY{n}{curr}\PY{p}{[}\PY{l+s+s1}{\PYZsq{}}\PY{l+s+s1}{y}\PY{l+s+s1}{\PYZsq{}}\PY{p}{]}\PY{p}{,} \PY{n}{BVZero}\PY{p}{(}\PY{n}{n}\PY{p}{)}\PY{p}{)}\PY{p}{,} 
            \PY{n}{Equals}\PY{p}{(}\PY{n}{BVZero}\PY{p}{(}\PY{l+m+mi}{1}\PY{p}{)}\PY{p}{,} \PY{n}{BVExtract}\PY{p}{(}\PY{n}{curr}\PY{p}{[}\PY{l+s+s1}{\PYZsq{}}\PY{l+s+s1}{y}\PY{l+s+s1}{\PYZsq{}}\PY{p}{]}\PY{p}{,} \PY{n}{start}\PY{o}{=}\PY{l+m+mi}{0}\PY{p}{,} \PY{n}{end}\PY{o}{=}\PY{l+m+mi}{0}\PY{p}{)}\PY{p}{)}\PY{p}{,}
            \PY{n}{Equals}\PY{p}{(}\PY{n}{prox}\PY{p}{[}\PY{l+s+s1}{\PYZsq{}}\PY{l+s+s1}{x}\PY{l+s+s1}{\PYZsq{}}\PY{p}{]}\PY{p}{,} \PY{n}{BVMul}\PY{p}{(}\PY{n}{curr}\PY{p}{[}\PY{l+s+s1}{\PYZsq{}}\PY{l+s+s1}{x}\PY{l+s+s1}{\PYZsq{}}\PY{p}{]}\PY{p}{,} \PY{n}{BV}\PY{p}{(}\PY{l+m+mi}{2}\PY{p}{,} \PY{n}{n}\PY{p}{)}\PY{p}{)}\PY{p}{)}\PY{p}{,}
            \PY{n}{Equals}\PY{p}{(}\PY{n}{prox}\PY{p}{[}\PY{l+s+s1}{\PYZsq{}}\PY{l+s+s1}{y}\PY{l+s+s1}{\PYZsq{}}\PY{p}{]}\PY{p}{,} \PY{n}{BVUDiv}\PY{p}{(}\PY{n}{curr}\PY{p}{[}\PY{l+s+s1}{\PYZsq{}}\PY{l+s+s1}{y}\PY{l+s+s1}{\PYZsq{}}\PY{p}{]}\PY{p}{,} \PY{n}{BV}\PY{p}{(}\PY{l+m+mi}{2}\PY{p}{,} \PY{n}{n}\PY{p}{)}\PY{p}{)}\PY{p}{)}\PY{p}{,}
            \PY{n}{Equals}\PY{p}{(}\PY{n}{prox}\PY{p}{[}\PY{l+s+s1}{\PYZsq{}}\PY{l+s+s1}{z}\PY{l+s+s1}{\PYZsq{}}\PY{p}{]}\PY{p}{,} \PY{n}{curr}\PY{p}{[}\PY{l+s+s1}{\PYZsq{}}\PY{l+s+s1}{z}\PY{l+s+s1}{\PYZsq{}}\PY{p}{]}\PY{p}{)}\PY{p}{,}
            \PY{n}{NotEquals}\PY{p}{(}\PY{n}{BVUDiv}\PY{p}{(}\PY{n}{prox}\PY{p}{[}\PY{l+s+s1}{\PYZsq{}}\PY{l+s+s1}{x}\PY{l+s+s1}{\PYZsq{}}\PY{p}{]}\PY{p}{,} \PY{n}{curr}\PY{p}{[}\PY{l+s+s1}{\PYZsq{}}\PY{l+s+s1}{x}\PY{l+s+s1}{\PYZsq{}}\PY{p}{]}\PY{p}{)}\PY{p}{,} \PY{n}{BV}\PY{p}{(}\PY{l+m+mi}{2}\PY{p}{,} \PY{n}{n}\PY{p}{)}\PY{p}{)}\PY{p}{,}
        \PY{p}{)}
    
    \PY{n}{t4} \PY{o}{=} \PY{n}{And}\PY{p}{(}
            \PY{n}{Equals}\PY{p}{(}\PY{n}{curr}\PY{p}{[}\PY{l+s+s1}{\PYZsq{}}\PY{l+s+s1}{pc}\PY{l+s+s1}{\PYZsq{}}\PY{p}{]}\PY{p}{,} \PY{n}{Int}\PY{p}{(}\PY{l+m+mi}{2}\PY{p}{)}\PY{p}{)}\PY{p}{,}
            \PY{n}{Equals}\PY{p}{(}\PY{n}{prox}\PY{p}{[}\PY{l+s+s1}{\PYZsq{}}\PY{l+s+s1}{pc}\PY{l+s+s1}{\PYZsq{}}\PY{p}{]}\PY{p}{,} \PY{n}{Int}\PY{p}{(}\PY{l+m+mi}{1}\PY{p}{)}\PY{p}{)}\PY{p}{,}
            \PY{n}{Equals}\PY{p}{(}\PY{n}{prox}\PY{p}{[}\PY{l+s+s1}{\PYZsq{}}\PY{l+s+s1}{x}\PY{l+s+s1}{\PYZsq{}}\PY{p}{]}\PY{p}{,} \PY{n}{curr}\PY{p}{[}\PY{l+s+s1}{\PYZsq{}}\PY{l+s+s1}{x}\PY{l+s+s1}{\PYZsq{}}\PY{p}{]}\PY{p}{)}\PY{p}{,} 
            \PY{n}{Equals}\PY{p}{(}\PY{n}{prox}\PY{p}{[}\PY{l+s+s1}{\PYZsq{}}\PY{l+s+s1}{y}\PY{l+s+s1}{\PYZsq{}}\PY{p}{]}\PY{p}{,} \PY{n}{curr}\PY{p}{[}\PY{l+s+s1}{\PYZsq{}}\PY{l+s+s1}{y}\PY{l+s+s1}{\PYZsq{}}\PY{p}{]}\PY{p}{)}\PY{p}{,} 
            \PY{n}{Equals}\PY{p}{(}\PY{n}{prox}\PY{p}{[}\PY{l+s+s1}{\PYZsq{}}\PY{l+s+s1}{z}\PY{l+s+s1}{\PYZsq{}}\PY{p}{]}\PY{p}{,} \PY{n}{curr}\PY{p}{[}\PY{l+s+s1}{\PYZsq{}}\PY{l+s+s1}{z}\PY{l+s+s1}{\PYZsq{}}\PY{p}{]}\PY{p}{)} 
        \PY{p}{)}

    \PY{n}{t5} \PY{o}{=} \PY{n}{And}\PY{p}{(}
            \PY{n}{Equals}\PY{p}{(}\PY{n}{curr}\PY{p}{[}\PY{l+s+s1}{\PYZsq{}}\PY{l+s+s1}{pc}\PY{l+s+s1}{\PYZsq{}}\PY{p}{]}\PY{p}{,} \PY{n}{Int}\PY{p}{(}\PY{l+m+mi}{1}\PY{p}{)}\PY{p}{)}\PY{p}{,}
            \PY{n}{Equals}\PY{p}{(}\PY{n}{prox}\PY{p}{[}\PY{l+s+s1}{\PYZsq{}}\PY{l+s+s1}{pc}\PY{l+s+s1}{\PYZsq{}}\PY{p}{]}\PY{p}{,} \PY{n}{Int}\PY{p}{(}\PY{l+m+mi}{3}\PY{p}{)}\PY{p}{)}\PY{p}{,} 
            \PY{n}{NotEquals}\PY{p}{(}\PY{n}{curr}\PY{p}{[}\PY{l+s+s1}{\PYZsq{}}\PY{l+s+s1}{y}\PY{l+s+s1}{\PYZsq{}}\PY{p}{]}\PY{p}{,} \PY{n}{BVZero}\PY{p}{(}\PY{n}{n}\PY{p}{)}\PY{p}{)}\PY{p}{,} 
            \PY{n}{Equals}\PY{p}{(}\PY{n}{BVOne}\PY{p}{(}\PY{l+m+mi}{1}\PY{p}{)}\PY{p}{,} \PY{n}{BVExtract}\PY{p}{(}\PY{n}{curr}\PY{p}{[}\PY{l+s+s1}{\PYZsq{}}\PY{l+s+s1}{y}\PY{l+s+s1}{\PYZsq{}}\PY{p}{]}\PY{p}{,} \PY{n}{start}\PY{o}{=}\PY{l+m+mi}{0}\PY{p}{,} \PY{n}{end}\PY{o}{=}\PY{l+m+mi}{0}\PY{p}{)}\PY{p}{)}\PY{p}{,}
            \PY{n}{Equals}\PY{p}{(}\PY{n}{prox}\PY{p}{[}\PY{l+s+s1}{\PYZsq{}}\PY{l+s+s1}{x}\PY{l+s+s1}{\PYZsq{}}\PY{p}{]}\PY{p}{,} \PY{n}{curr}\PY{p}{[}\PY{l+s+s1}{\PYZsq{}}\PY{l+s+s1}{x}\PY{l+s+s1}{\PYZsq{}}\PY{p}{]}\PY{p}{)}\PY{p}{,}
            \PY{n}{Equals}\PY{p}{(}\PY{n}{prox}\PY{p}{[}\PY{l+s+s1}{\PYZsq{}}\PY{l+s+s1}{y}\PY{l+s+s1}{\PYZsq{}}\PY{p}{]}\PY{p}{,} \PY{n}{BVSub}\PY{p}{(}\PY{n}{curr}\PY{p}{[}\PY{l+s+s1}{\PYZsq{}}\PY{l+s+s1}{y}\PY{l+s+s1}{\PYZsq{}}\PY{p}{]}\PY{p}{,} \PY{n}{BV}\PY{p}{(}\PY{l+m+mi}{1}\PY{p}{,} \PY{n}{n}\PY{p}{)}\PY{p}{)}\PY{p}{)}\PY{p}{,}
            \PY{n}{Equals}\PY{p}{(}\PY{n}{prox}\PY{p}{[}\PY{l+s+s1}{\PYZsq{}}\PY{l+s+s1}{z}\PY{l+s+s1}{\PYZsq{}}\PY{p}{]}\PY{p}{,} \PY{n}{BVAdd}\PY{p}{(}\PY{n}{curr}\PY{p}{[}\PY{l+s+s1}{\PYZsq{}}\PY{l+s+s1}{z}\PY{l+s+s1}{\PYZsq{}}\PY{p}{]}\PY{p}{,} \PY{n}{curr}\PY{p}{[}\PY{l+s+s1}{\PYZsq{}}\PY{l+s+s1}{x}\PY{l+s+s1}{\PYZsq{}}\PY{p}{]}\PY{p}{)}\PY{p}{)}\PY{p}{,}
            \PY{n}{BVUGE}\PY{p}{(}\PY{n}{prox}\PY{p}{[}\PY{l+s+s1}{\PYZsq{}}\PY{l+s+s1}{z}\PY{l+s+s1}{\PYZsq{}}\PY{p}{]}\PY{p}{,} \PY{n}{curr}\PY{p}{[}\PY{l+s+s1}{\PYZsq{}}\PY{l+s+s1}{z}\PY{l+s+s1}{\PYZsq{}}\PY{p}{]}\PY{p}{)} 
        \PY{p}{)}

    \PY{n}{t6} \PY{o}{=} \PY{n}{And}\PY{p}{(}
            \PY{n}{Equals}\PY{p}{(}\PY{n}{curr}\PY{p}{[}\PY{l+s+s1}{\PYZsq{}}\PY{l+s+s1}{pc}\PY{l+s+s1}{\PYZsq{}}\PY{p}{]}\PY{p}{,} \PY{n}{Int}\PY{p}{(}\PY{l+m+mi}{1}\PY{p}{)}\PY{p}{)}\PY{p}{,}
            \PY{n}{Equals}\PY{p}{(}\PY{n}{prox}\PY{p}{[}\PY{l+s+s1}{\PYZsq{}}\PY{l+s+s1}{pc}\PY{l+s+s1}{\PYZsq{}}\PY{p}{]}\PY{p}{,} \PY{n}{Int}\PY{p}{(}\PY{o}{\PYZhy{}}\PY{l+m+mi}{1}\PY{p}{)}\PY{p}{)}\PY{p}{,} 
            \PY{n}{NotEquals}\PY{p}{(}\PY{n}{curr}\PY{p}{[}\PY{l+s+s1}{\PYZsq{}}\PY{l+s+s1}{y}\PY{l+s+s1}{\PYZsq{}}\PY{p}{]}\PY{p}{,} \PY{n}{BVZero}\PY{p}{(}\PY{n}{n}\PY{p}{)}\PY{p}{)}\PY{p}{,} 
            \PY{n}{Equals}\PY{p}{(}\PY{n}{BVOne}\PY{p}{(}\PY{l+m+mi}{1}\PY{p}{)}\PY{p}{,} \PY{n}{BVExtract}\PY{p}{(}\PY{n}{curr}\PY{p}{[}\PY{l+s+s1}{\PYZsq{}}\PY{l+s+s1}{y}\PY{l+s+s1}{\PYZsq{}}\PY{p}{]}\PY{p}{,} \PY{n}{start}\PY{o}{=}\PY{l+m+mi}{0}\PY{p}{,} \PY{n}{end}\PY{o}{=}\PY{l+m+mi}{0}\PY{p}{)}\PY{p}{)}\PY{p}{,}
            \PY{n}{Equals}\PY{p}{(}\PY{n}{prox}\PY{p}{[}\PY{l+s+s1}{\PYZsq{}}\PY{l+s+s1}{x}\PY{l+s+s1}{\PYZsq{}}\PY{p}{]}\PY{p}{,} \PY{n}{curr}\PY{p}{[}\PY{l+s+s1}{\PYZsq{}}\PY{l+s+s1}{x}\PY{l+s+s1}{\PYZsq{}}\PY{p}{]}\PY{p}{)}\PY{p}{,}
            \PY{n}{Equals}\PY{p}{(}\PY{n}{prox}\PY{p}{[}\PY{l+s+s1}{\PYZsq{}}\PY{l+s+s1}{y}\PY{l+s+s1}{\PYZsq{}}\PY{p}{]}\PY{p}{,} \PY{n}{BVSub}\PY{p}{(}\PY{n}{curr}\PY{p}{[}\PY{l+s+s1}{\PYZsq{}}\PY{l+s+s1}{y}\PY{l+s+s1}{\PYZsq{}}\PY{p}{]}\PY{p}{,} \PY{n}{BV}\PY{p}{(}\PY{l+m+mi}{1}\PY{p}{,} \PY{n}{n}\PY{p}{)}\PY{p}{)}\PY{p}{)}\PY{p}{,}
            \PY{n}{Equals}\PY{p}{(}\PY{n}{prox}\PY{p}{[}\PY{l+s+s1}{\PYZsq{}}\PY{l+s+s1}{z}\PY{l+s+s1}{\PYZsq{}}\PY{p}{]}\PY{p}{,} \PY{n}{BVAdd}\PY{p}{(}\PY{n}{curr}\PY{p}{[}\PY{l+s+s1}{\PYZsq{}}\PY{l+s+s1}{z}\PY{l+s+s1}{\PYZsq{}}\PY{p}{]}\PY{p}{,} \PY{n}{curr}\PY{p}{[}\PY{l+s+s1}{\PYZsq{}}\PY{l+s+s1}{x}\PY{l+s+s1}{\PYZsq{}}\PY{p}{]}\PY{p}{)}\PY{p}{)}\PY{p}{,}
            \PY{n}{BVULE}\PY{p}{(}\PY{n}{prox}\PY{p}{[}\PY{l+s+s1}{\PYZsq{}}\PY{l+s+s1}{z}\PY{l+s+s1}{\PYZsq{}}\PY{p}{]}\PY{p}{,} \PY{n}{curr}\PY{p}{[}\PY{l+s+s1}{\PYZsq{}}\PY{l+s+s1}{z}\PY{l+s+s1}{\PYZsq{}}\PY{p}{]}\PY{p}{)}
        \PY{p}{)}

    \PY{n}{t7} \PY{o}{=} \PY{n}{And}\PY{p}{(}
            \PY{n}{Equals}\PY{p}{(}\PY{n}{curr}\PY{p}{[}\PY{l+s+s1}{\PYZsq{}}\PY{l+s+s1}{pc}\PY{l+s+s1}{\PYZsq{}}\PY{p}{]}\PY{p}{,} \PY{n}{Int}\PY{p}{(}\PY{l+m+mi}{3}\PY{p}{)}\PY{p}{)}\PY{p}{,}
            \PY{n}{Equals}\PY{p}{(}\PY{n}{prox}\PY{p}{[}\PY{l+s+s1}{\PYZsq{}}\PY{l+s+s1}{pc}\PY{l+s+s1}{\PYZsq{}}\PY{p}{]}\PY{p}{,} \PY{n}{Int}\PY{p}{(}\PY{l+m+mi}{1}\PY{p}{)}\PY{p}{)}\PY{p}{,}
            \PY{n}{Equals}\PY{p}{(}\PY{n}{prox}\PY{p}{[}\PY{l+s+s1}{\PYZsq{}}\PY{l+s+s1}{x}\PY{l+s+s1}{\PYZsq{}}\PY{p}{]}\PY{p}{,} \PY{n}{curr}\PY{p}{[}\PY{l+s+s1}{\PYZsq{}}\PY{l+s+s1}{x}\PY{l+s+s1}{\PYZsq{}}\PY{p}{]}\PY{p}{)}\PY{p}{,} 
            \PY{n}{Equals}\PY{p}{(}\PY{n}{prox}\PY{p}{[}\PY{l+s+s1}{\PYZsq{}}\PY{l+s+s1}{y}\PY{l+s+s1}{\PYZsq{}}\PY{p}{]}\PY{p}{,} \PY{n}{curr}\PY{p}{[}\PY{l+s+s1}{\PYZsq{}}\PY{l+s+s1}{y}\PY{l+s+s1}{\PYZsq{}}\PY{p}{]}\PY{p}{)}\PY{p}{,} 
            \PY{n}{Equals}\PY{p}{(}\PY{n}{prox}\PY{p}{[}\PY{l+s+s1}{\PYZsq{}}\PY{l+s+s1}{z}\PY{l+s+s1}{\PYZsq{}}\PY{p}{]}\PY{p}{,} \PY{n}{curr}\PY{p}{[}\PY{l+s+s1}{\PYZsq{}}\PY{l+s+s1}{z}\PY{l+s+s1}{\PYZsq{}}\PY{p}{]}\PY{p}{)}\PY{p}{,}
        \PY{p}{)}

    \PY{n}{t8} \PY{o}{=} \PY{n}{And}\PY{p}{(}
            \PY{n}{Equals}\PY{p}{(}\PY{n}{curr}\PY{p}{[}\PY{l+s+s1}{\PYZsq{}}\PY{l+s+s1}{pc}\PY{l+s+s1}{\PYZsq{}}\PY{p}{]}\PY{p}{,} \PY{n}{Int}\PY{p}{(}\PY{l+m+mi}{1}\PY{p}{)}\PY{p}{)}\PY{p}{,}
            \PY{n}{Equals}\PY{p}{(}\PY{n}{prox}\PY{p}{[}\PY{l+s+s1}{\PYZsq{}}\PY{l+s+s1}{pc}\PY{l+s+s1}{\PYZsq{}}\PY{p}{]}\PY{p}{,} \PY{n}{Int}\PY{p}{(}\PY{l+m+mi}{4}\PY{p}{)}\PY{p}{)}\PY{p}{,}
            \PY{n}{Equals}\PY{p}{(}\PY{n}{curr}\PY{p}{[}\PY{l+s+s1}{\PYZsq{}}\PY{l+s+s1}{y}\PY{l+s+s1}{\PYZsq{}}\PY{p}{]}\PY{p}{,} \PY{n}{BVZero}\PY{p}{(}\PY{n}{n}\PY{p}{)}\PY{p}{)}\PY{p}{,}
            \PY{n}{Equals}\PY{p}{(}\PY{n}{prox}\PY{p}{[}\PY{l+s+s1}{\PYZsq{}}\PY{l+s+s1}{x}\PY{l+s+s1}{\PYZsq{}}\PY{p}{]}\PY{p}{,} \PY{n}{curr}\PY{p}{[}\PY{l+s+s1}{\PYZsq{}}\PY{l+s+s1}{x}\PY{l+s+s1}{\PYZsq{}}\PY{p}{]}\PY{p}{)}\PY{p}{,} 
            \PY{n}{Equals}\PY{p}{(}\PY{n}{prox}\PY{p}{[}\PY{l+s+s1}{\PYZsq{}}\PY{l+s+s1}{y}\PY{l+s+s1}{\PYZsq{}}\PY{p}{]}\PY{p}{,} \PY{n}{curr}\PY{p}{[}\PY{l+s+s1}{\PYZsq{}}\PY{l+s+s1}{y}\PY{l+s+s1}{\PYZsq{}}\PY{p}{]}\PY{p}{)}\PY{p}{,} 
            \PY{n}{Equals}\PY{p}{(}\PY{n}{prox}\PY{p}{[}\PY{l+s+s1}{\PYZsq{}}\PY{l+s+s1}{z}\PY{l+s+s1}{\PYZsq{}}\PY{p}{]}\PY{p}{,} \PY{n}{curr}\PY{p}{[}\PY{l+s+s1}{\PYZsq{}}\PY{l+s+s1}{z}\PY{l+s+s1}{\PYZsq{}}\PY{p}{]}\PY{p}{)} 
    \PY{p}{)}

    \PY{n}{tSTOP} \PY{o}{=} \PY{n}{And}\PY{p}{(}
                \PY{n}{Equals}\PY{p}{(}\PY{n}{curr}\PY{p}{[}\PY{l+s+s1}{\PYZsq{}}\PY{l+s+s1}{pc}\PY{l+s+s1}{\PYZsq{}}\PY{p}{]}\PY{p}{,} \PY{n}{Int}\PY{p}{(}\PY{l+m+mi}{4}\PY{p}{)}\PY{p}{)}\PY{p}{,}
                \PY{n}{Equals}\PY{p}{(}\PY{n}{prox}\PY{p}{[}\PY{l+s+s1}{\PYZsq{}}\PY{l+s+s1}{pc}\PY{l+s+s1}{\PYZsq{}}\PY{p}{]}\PY{p}{,} \PY{n}{Int}\PY{p}{(}\PY{l+m+mi}{4}\PY{p}{)}\PY{p}{)}\PY{p}{,}
                \PY{n}{Equals}\PY{p}{(}\PY{n}{prox}\PY{p}{[}\PY{l+s+s1}{\PYZsq{}}\PY{l+s+s1}{x}\PY{l+s+s1}{\PYZsq{}}\PY{p}{]}\PY{p}{,} \PY{n}{curr}\PY{p}{[}\PY{l+s+s1}{\PYZsq{}}\PY{l+s+s1}{x}\PY{l+s+s1}{\PYZsq{}}\PY{p}{]}\PY{p}{)}\PY{p}{,} 
                \PY{n}{Equals}\PY{p}{(}\PY{n}{prox}\PY{p}{[}\PY{l+s+s1}{\PYZsq{}}\PY{l+s+s1}{y}\PY{l+s+s1}{\PYZsq{}}\PY{p}{]}\PY{p}{,} \PY{n}{curr}\PY{p}{[}\PY{l+s+s1}{\PYZsq{}}\PY{l+s+s1}{y}\PY{l+s+s1}{\PYZsq{}}\PY{p}{]}\PY{p}{)}\PY{p}{,} 
                \PY{n}{Equals}\PY{p}{(}\PY{n}{prox}\PY{p}{[}\PY{l+s+s1}{\PYZsq{}}\PY{l+s+s1}{z}\PY{l+s+s1}{\PYZsq{}}\PY{p}{]}\PY{p}{,} \PY{n}{curr}\PY{p}{[}\PY{l+s+s1}{\PYZsq{}}\PY{l+s+s1}{z}\PY{l+s+s1}{\PYZsq{}}\PY{p}{]}\PY{p}{)} 
            \PY{p}{)}


    \PY{n}{tERROR} \PY{o}{=} \PY{n}{And}\PY{p}{(}
                \PY{n}{Equals}\PY{p}{(}\PY{n}{curr}\PY{p}{[}\PY{l+s+s1}{\PYZsq{}}\PY{l+s+s1}{pc}\PY{l+s+s1}{\PYZsq{}}\PY{p}{]}\PY{p}{,} \PY{n}{Int}\PY{p}{(}\PY{o}{\PYZhy{}}\PY{l+m+mi}{1}\PY{p}{)}\PY{p}{)}\PY{p}{,}
                \PY{n}{Equals}\PY{p}{(}\PY{n}{prox}\PY{p}{[}\PY{l+s+s1}{\PYZsq{}}\PY{l+s+s1}{pc}\PY{l+s+s1}{\PYZsq{}}\PY{p}{]}\PY{p}{,} \PY{n}{Int}\PY{p}{(}\PY{o}{\PYZhy{}}\PY{l+m+mi}{1}\PY{p}{)}\PY{p}{)}\PY{p}{,}
                \PY{n}{Equals}\PY{p}{(}\PY{n}{prox}\PY{p}{[}\PY{l+s+s1}{\PYZsq{}}\PY{l+s+s1}{x}\PY{l+s+s1}{\PYZsq{}}\PY{p}{]}\PY{p}{,} \PY{n}{curr}\PY{p}{[}\PY{l+s+s1}{\PYZsq{}}\PY{l+s+s1}{x}\PY{l+s+s1}{\PYZsq{}}\PY{p}{]}\PY{p}{)}\PY{p}{,} 
                \PY{n}{Equals}\PY{p}{(}\PY{n}{prox}\PY{p}{[}\PY{l+s+s1}{\PYZsq{}}\PY{l+s+s1}{y}\PY{l+s+s1}{\PYZsq{}}\PY{p}{]}\PY{p}{,} \PY{n}{curr}\PY{p}{[}\PY{l+s+s1}{\PYZsq{}}\PY{l+s+s1}{y}\PY{l+s+s1}{\PYZsq{}}\PY{p}{]}\PY{p}{)}\PY{p}{,} 
                \PY{n}{Equals}\PY{p}{(}\PY{n}{prox}\PY{p}{[}\PY{l+s+s1}{\PYZsq{}}\PY{l+s+s1}{z}\PY{l+s+s1}{\PYZsq{}}\PY{p}{]}\PY{p}{,} \PY{n}{curr}\PY{p}{[}\PY{l+s+s1}{\PYZsq{}}\PY{l+s+s1}{z}\PY{l+s+s1}{\PYZsq{}}\PY{p}{]}\PY{p}{)} 
            \PY{p}{)}

    \PY{k}{return} \PY{n}{Or}\PY{p}{(}\PY{n}{t1}\PY{p}{,} \PY{n}{t2}\PY{p}{,} \PY{n}{t3}\PY{p}{,} \PY{n}{t4}\PY{p}{,} \PY{n}{t5}\PY{p}{,} \PY{n}{t6}\PY{p}{,} \PY{n}{t7}\PY{p}{,} \PY{n}{t8}\PY{p}{,} \PY{n}{tSTOP}\PY{p}{,} \PY{n}{tERROR}\PY{p}{)}
\end{Verbatim}
\end{tcolorbox}

    \hypertarget{funuxe7uxe3o-gera_traco}{%
    \textbf{\large Função gera\_traco}\label{funuxe7uxe3o-gera_traco}}

Dada uma função que gera uma cópia das variáveis do estado, um predicado
que testa se um estado é inicial, um predicado que testa se um par de
estados é uma transição válida, e um número positivo k, utilizamos o SMT
solver para gerar um possível traço de execução do programa de tamanho
k. Para cada estado do traço imprimimos o respectivo valor das
variáveis.

    \begin{tcolorbox}[breakable, size=fbox, boxrule=1pt, pad at break*=1mm,colback=cellbackground, colframe=cellborder]
\prompt{In}{incolor}{6}{\boxspacing}
\begin{Verbatim}[commandchars=\\\{\}]
\PY{k}{def} \PY{n+nf}{gera\PYZus{}traco}\PY{p}{(}\PY{n}{declare}\PY{p}{,}\PY{n}{init}\PY{p}{,}\PY{n}{trans}\PY{p}{,}\PY{n}{k}\PY{p}{)}\PY{p}{:}

    \PY{k}{with} \PY{n}{Solver}\PY{p}{(}\PY{n}{name}\PY{o}{=}\PY{l+s+s2}{\PYZdq{}}\PY{l+s+s2}{z3}\PY{l+s+s2}{\PYZdq{}}\PY{p}{)} \PY{k}{as} \PY{n}{s}\PY{p}{:}
        \PY{n}{trace} \PY{o}{=} \PY{p}{[}\PY{n}{declare}\PY{p}{(}\PY{n}{i}\PY{p}{)} \PY{k}{for} \PY{n}{i} \PY{o+ow}{in} \PY{n+nb}{range}\PY{p}{(}\PY{n}{k}\PY{p}{)}\PY{p}{]}
        
        \PY{c+c1}{\PYZsh{}adicionar o estado inicial}
        \PY{n}{s}\PY{o}{.}\PY{n}{add\PYZus{}assertion}\PY{p}{(}\PY{n}{init}\PY{p}{(}\PY{n}{trace}\PY{p}{[}\PY{l+m+mi}{0}\PY{p}{]}\PY{p}{)}\PY{p}{)}
        
        \PY{c+c1}{\PYZsh{}adicionar a função ed transição}
        \PY{k}{for} \PY{n}{i} \PY{o+ow}{in} \PY{n+nb}{range}\PY{p}{(}\PY{n}{k}\PY{o}{\PYZhy{}}\PY{l+m+mi}{1}\PY{p}{)}\PY{p}{:}
            \PY{n}{s}\PY{o}{.}\PY{n}{add\PYZus{}assertion}\PY{p}{(}\PY{n}{trans}\PY{p}{(}\PY{n}{trace}\PY{p}{[}\PY{n}{i}\PY{p}{]}\PY{p}{,} \PY{n}{trace}\PY{p}{[}\PY{n}{i}\PY{o}{+}\PY{l+m+mi}{1}\PY{p}{]}\PY{p}{)}\PY{p}{)}
        
        \PY{k}{if} \PY{n}{s}\PY{o}{.}\PY{n}{solve}\PY{p}{(}\PY{p}{)}\PY{p}{:}
            \PY{k}{for} \PY{n}{i} \PY{o+ow}{in} \PY{n+nb}{range}\PY{p}{(}\PY{n}{k}\PY{p}{)}\PY{p}{:}
                \PY{n+nb}{print}\PY{p}{(}\PY{l+s+s2}{\PYZdq{}}\PY{l+s+s2}{Passo}\PY{l+s+s2}{\PYZdq{}}\PY{p}{,} \PY{n}{i}\PY{p}{)}
                \PY{k}{for} \PY{n}{v} \PY{o+ow}{in} \PY{n}{trace}\PY{p}{[}\PY{n}{i}\PY{p}{]}\PY{p}{:}
                    \PY{n+nb}{print}\PY{p}{(}\PY{n}{v}\PY{p}{,} \PY{l+s+s2}{\PYZdq{}}\PY{l+s+s2}{=}\PY{l+s+s2}{\PYZdq{}}\PY{p}{,} \PY{n}{s}\PY{o}{.}\PY{n}{get\PYZus{}value}\PY{p}{(}\PY{n}{trace}\PY{p}{[}\PY{n}{i}\PY{p}{]}\PY{p}{[}\PY{n}{v}\PY{p}{]}\PY{p}{)}\PY{p}{)}
                \PY{n+nb}{print}\PY{p}{(}\PY{l+s+s2}{\PYZdq{}}\PY{l+s+s2}{\PYZhy{}\PYZhy{}\PYZhy{}\PYZhy{}\PYZhy{}\PYZhy{}\PYZhy{}\PYZhy{}\PYZhy{}\PYZhy{}\PYZhy{}\PYZhy{}\PYZhy{}\PYZhy{}}\PY{l+s+s2}{\PYZdq{}}\PY{p}{)}

\PY{c+c1}{\PYZsh{} gera\PYZus{}traco(declare,init,trans,20)}
\end{Verbatim}
\end{tcolorbox}

    \hypertarget{funuxe7uxe3o-inv}{%
    \textbf{\large Função inv}\label{funuxe7uxe3o-inv}}

Dado um possível estado do programa (um dicionário de variáveis),
devolve um predicado do pySMT que testa se nesse estado \textbf{x} *
\textbf{y} + \textbf{z} = \textbf{a} * \textbf{b}, que é o invariante
que queremos verificar

    \begin{tcolorbox}[breakable, size=fbox, boxrule=1pt, pad at break*=1mm,colback=cellbackground, colframe=cellborder]
\prompt{In}{incolor}{7}{\boxspacing}
\begin{Verbatim}[commandchars=\\\{\}]
\PY{k}{def} \PY{n+nf}{inv}\PY{p}{(}\PY{n}{state}\PY{p}{)}\PY{p}{:}
    \PY{k}{return} \PY{n}{Equals}\PY{p}{(}\PY{n}{BVAdd}\PY{p}{(}\PY{n}{BVMul}\PY{p}{(}\PY{n}{state}\PY{p}{[}\PY{l+s+s1}{\PYZsq{}}\PY{l+s+s1}{x}\PY{l+s+s1}{\PYZsq{}}\PY{p}{]}\PY{p}{,} \PY{n}{state}\PY{p}{[}\PY{l+s+s1}{\PYZsq{}}\PY{l+s+s1}{y}\PY{l+s+s1}{\PYZsq{}}\PY{p}{]}\PY{p}{)}\PY{p}{,} \PY{n}{state}\PY{p}{[}\PY{l+s+s1}{\PYZsq{}}\PY{l+s+s1}{z}\PY{l+s+s1}{\PYZsq{}}\PY{p}{]}\PY{p}{)}\PY{p}{,} \PY{n}{BVMul}\PY{p}{(}\PY{n}{BV}\PY{p}{(}\PY{n}{a}\PY{p}{,} \PY{n}{n}\PY{p}{)}\PY{p}{,} \PY{n}{BV}\PY{p}{(}\PY{n}{b}\PY{p}{,} \PY{n}{n}\PY{p}{)}\PY{p}{)}\PY{p}{)}
\end{Verbatim}
\end{tcolorbox}

    \hypertarget{verificauxe7uxe3o-indutiva-de-invariantes}{%
    \textbf{\large Verificação indutiva de
invariantes}\label{verificauxe7uxe3o-indutiva-de-invariantes}}

No caso da verificação de propriedades de segurança \(G\ \phi\), para
verificar o invariante \(\phi\) por indução temos que verificar as
seguintes condições: - \(\phi\) é válido nos estados iniciais, ou seja,
\(\mathit{init}(s) \rightarrow \phi(s)\) - Para qualquer estado,
assumindo que \(\phi\) é verdade, se executarmos uma transição, \(\phi\)
continua a ser verdade no próximo estado, ou seja,
\(\phi(s) \wedge \mathit{trans}(s,s') \rightarrow \phi(s')\).

    \hypertarget{funuxe7uxe3o-induction_always}{%
    \textbf{\large Função
induction\_always}\label{funuxe7uxe3o-induction_always}}

Verifica invariantes por indução. A função recebe como argumento uma
função que gera uma cópia das variáveis do estado, um predicado que
testa se um estado é inicial, um predicado que testa se um par de
estados é uma transição válida, e o invariante.

Teremos que testar a validade das duas condições acima recorrendo à
satisfiabilidade, ou seja, usando o solver para encontrar
contra-exemplos, devendo o procedimento reportar qual das propriedades
falha. Por exemplo, no caso da primeira deve procurar uma valoração que
satisfaça \(\mathit{init}(s) \wedge \neg \phi(s)\).

    \begin{tcolorbox}[breakable, size=fbox, boxrule=1pt, pad at break*=1mm,colback=cellbackground, colframe=cellborder]
\prompt{In}{incolor}{8}{\boxspacing}
\begin{Verbatim}[commandchars=\\\{\}]
\PY{k}{def} \PY{n+nf}{induction\PYZus{}always}\PY{p}{(}\PY{n}{declare}\PY{p}{,}\PY{n}{init}\PY{p}{,}\PY{n}{trans}\PY{p}{,}\PY{n}{inv}\PY{p}{)}\PY{p}{:}
    \PY{k}{with} \PY{n}{Solver}\PY{p}{(}\PY{n}{name}\PY{o}{=}\PY{l+s+s2}{\PYZdq{}}\PY{l+s+s2}{z3}\PY{l+s+s2}{\PYZdq{}}\PY{p}{)} \PY{k}{as} \PY{n}{s}\PY{p}{:}
        \PY{n}{s\PYZus{}now} \PY{o}{=} \PY{n}{declare}\PY{p}{(}\PY{l+m+mi}{0}\PY{p}{)}
        \PY{n}{s\PYZus{}next} \PY{o}{=} \PY{n}{declare}\PY{p}{(}\PY{l+m+mi}{1}\PY{p}{)}
        
        \PY{c+c1}{\PYZsh{}caso base}
        \PY{n}{s}\PY{o}{.}\PY{n}{push}\PY{p}{(}\PY{p}{)}
        \PY{n}{s}\PY{o}{.}\PY{n}{add\PYZus{}assertion}\PY{p}{(}\PY{n}{init}\PY{p}{(}\PY{n}{s\PYZus{}now}\PY{p}{)}\PY{p}{)}
        \PY{n}{s}\PY{o}{.}\PY{n}{add\PYZus{}assertion}\PY{p}{(}\PY{n}{Not}\PY{p}{(}\PY{n}{inv}\PY{p}{(}\PY{n}{s\PYZus{}now}\PY{p}{)}\PY{p}{)}\PY{p}{)}
        
        \PY{k}{if} \PY{n}{s}\PY{o}{.}\PY{n}{solve}\PY{p}{(}\PY{p}{)}\PY{p}{:}
            \PY{c+c1}{\PYZsh{}significa que encontramos um contraexemplo}
            \PY{n+nb}{print}\PY{p}{(}\PY{l+s+s2}{\PYZdq{}}\PY{l+s+s2}{A propriedade não é válida}\PY{l+s+s2}{\PYZdq{}}\PY{p}{)}
            \PY{k}{return}
        \PY{n}{s}\PY{o}{.}\PY{n}{pop}\PY{p}{(}\PY{p}{)} \PY{c+c1}{\PYZsh{}limpa tudo o que foi posto depois do push no solver}
        
        \PY{c+c1}{\PYZsh{}passo de indução}
        \PY{n}{s}\PY{o}{.}\PY{n}{push}\PY{p}{(}\PY{p}{)}
        \PY{n}{s}\PY{o}{.}\PY{n}{add\PYZus{}assertion}\PY{p}{(}\PY{n}{inv}\PY{p}{(}\PY{n}{s\PYZus{}now}\PY{p}{)}\PY{p}{)}
        \PY{n}{s}\PY{o}{.}\PY{n}{add\PYZus{}assertion}\PY{p}{(}\PY{n}{trans}\PY{p}{(}\PY{n}{s\PYZus{}now}\PY{p}{,} \PY{n}{s\PYZus{}next}\PY{p}{)}\PY{p}{)}
        \PY{n}{s}\PY{o}{.}\PY{n}{add\PYZus{}assertion}\PY{p}{(}\PY{n}{Not}\PY{p}{(}\PY{n}{inv}\PY{p}{(}\PY{n}{s\PYZus{}next}\PY{p}{)}\PY{p}{)}\PY{p}{)}
        
        \PY{k}{if} \PY{n}{s}\PY{o}{.}\PY{n}{solve}\PY{p}{(}\PY{p}{)}\PY{p}{:}
            \PY{n+nb}{print}\PY{p}{(}\PY{l+s+s2}{\PYZdq{}}\PY{l+s+s2}{A propriedade não é válida}\PY{l+s+s2}{\PYZdq{}}\PY{p}{)}
            \PY{k}{for} \PY{n}{k} \PY{o+ow}{in} \PY{n}{s\PYZus{}now}\PY{p}{:}
                \PY{n+nb}{print}\PY{p}{(}\PY{n}{k}\PY{p}{,} \PY{l+s+s2}{\PYZdq{}}\PY{l+s+s2}{=}\PY{l+s+s2}{\PYZdq{}}\PY{p}{,} \PY{n}{s}\PY{o}{.}\PY{n}{get\PYZus{}value}\PY{p}{(}\PY{n}{s\PYZus{}now}\PY{p}{[}\PY{n}{k}\PY{p}{]}\PY{p}{)}\PY{p}{)}
            \PY{k}{return}
        \PY{n}{s}\PY{o}{.}\PY{n}{pop}\PY{p}{(}\PY{p}{)}

\PY{c+c1}{\PYZsh{} induction\PYZus{}always(declare, init, trans, inv)}
\end{Verbatim}
\end{tcolorbox}

    \hypertarget{verificauxe7uxe3o-do-invariante-x-y-z-a-b}{%
    \textbf{\large Verificação do invariante x * y + z = a *
b}\label{verificauxe7uxe3o-do-invariante-x-y-z-a-b}}

Como não encontramos contra-exemplos com o caso base e o passo de
indução, então o invariante verifica-se


    % Add a bibliography block to the postdoc
    
    
    
\end{document}
