\documentclass[11pt]{article}

    \usepackage[breakable]{tcolorbox}
    \usepackage{parskip} % Stop auto-indenting (to mimic markdown behaviour)
    

    % Basic figure setup, for now with no caption control since it's done
    % automatically by Pandoc (which extracts ![](path) syntax from Markdown).
    \usepackage{graphicx}
    % Maintain compatibility with old templates. Remove in nbconvert 6.0
    \let\Oldincludegraphics\includegraphics
    % Ensure that by default, figures have no caption (until we provide a
    % proper Figure object with a Caption API and a way to capture that
    % in the conversion process - todo).
    \usepackage{caption}
    \DeclareCaptionFormat{nocaption}{}
    \captionsetup{format=nocaption,aboveskip=0pt,belowskip=0pt}

    \usepackage{float}
    \floatplacement{figure}{H} % forces figures to be placed at the correct location
    \usepackage{xcolor} % Allow colors to be defined
    \usepackage{enumerate} % Needed for markdown enumerations to work
    \usepackage{geometry} % Used to adjust the document margins
    \usepackage{amsmath} % Equations
    \usepackage{amssymb} % Equations
    \usepackage{textcomp} % defines textquotesingle
    % Hack from http://tex.stackexchange.com/a/47451/13684:
    \AtBeginDocument{%
        \def\PYZsq{\textquotesingle}% Upright quotes in Pygmentized code
    }
    \usepackage{upquote} % Upright quotes for verbatim code
    \usepackage{eurosym} % defines \euro

    \usepackage{iftex}
    \ifPDFTeX
        \usepackage[T1]{fontenc}
        \IfFileExists{alphabeta.sty}{
              \usepackage{alphabeta}
          }{
              \usepackage[mathletters]{ucs}
              \usepackage[utf8x]{inputenc}
          }
    \else
        \usepackage{fontspec}
        \usepackage{unicode-math}
    \fi

    \usepackage{fancyvrb} % verbatim replacement that allows latex
    \usepackage{grffile} % extends the file name processing of package graphics 
                         % to support a larger range
    \makeatletter % fix for old versions of grffile with XeLaTeX
    \@ifpackagelater{grffile}{2019/11/01}
    {
      % Do nothing on new versions
    }
    {
      \def\Gread@@xetex#1{%
        \IfFileExists{"\Gin@base".bb}%
        {\Gread@eps{\Gin@base.bb}}%
        {\Gread@@xetex@aux#1}%
      }
    }
    \makeatother
    \usepackage[Export]{adjustbox} % Used to constrain images to a maximum size
    \adjustboxset{max size={0.9\linewidth}{0.9\paperheight}}

    % The hyperref package gives us a pdf with properly built
    % internal navigation ('pdf bookmarks' for the table of contents,
    % internal cross-reference links, web links for URLs, etc.)
    \usepackage{hyperref}
    % The default LaTeX title has an obnoxious amount of whitespace. By default,
    % titling removes some of it. It also provides customization options.
    \usepackage{titling}
    \usepackage{longtable} % longtable support required by pandoc >1.10
    \usepackage{booktabs}  % table support for pandoc > 1.12.2
    \usepackage{array}     % table support for pandoc >= 2.11.3
    \usepackage{calc}      % table minipage width calculation for pandoc >= 2.11.1
    \usepackage[inline]{enumitem} % IRkernel/repr support (it uses the enumerate* environment)
    \usepackage[normalem]{ulem} % ulem is needed to support strikethroughs (\sout)
                                % normalem makes italics be italics, not underlines
    \usepackage{mathrsfs}
    

    
    % Colors for the hyperref package
    \definecolor{urlcolor}{rgb}{0,.145,.698}
    \definecolor{linkcolor}{rgb}{.71,0.21,0.01}
    \definecolor{citecolor}{rgb}{.12,.54,.11}

    % ANSI colors
    \definecolor{ansi-black}{HTML}{3E424D}
    \definecolor{ansi-black-intense}{HTML}{282C36}
    \definecolor{ansi-red}{HTML}{E75C58}
    \definecolor{ansi-red-intense}{HTML}{B22B31}
    \definecolor{ansi-green}{HTML}{00A250}
    \definecolor{ansi-green-intense}{HTML}{007427}
    \definecolor{ansi-yellow}{HTML}{DDB62B}
    \definecolor{ansi-yellow-intense}{HTML}{B27D12}
    \definecolor{ansi-blue}{HTML}{208FFB}
    \definecolor{ansi-blue-intense}{HTML}{0065CA}
    \definecolor{ansi-magenta}{HTML}{D160C4}
    \definecolor{ansi-magenta-intense}{HTML}{A03196}
    \definecolor{ansi-cyan}{HTML}{60C6C8}
    \definecolor{ansi-cyan-intense}{HTML}{258F8F}
    \definecolor{ansi-white}{HTML}{C5C1B4}
    \definecolor{ansi-white-intense}{HTML}{A1A6B2}
    \definecolor{ansi-default-inverse-fg}{HTML}{FFFFFF}
    \definecolor{ansi-default-inverse-bg}{HTML}{000000}

    % common color for the border for error outputs.
    \definecolor{outerrorbackground}{HTML}{FFDFDF}

    % commands and environments needed by pandoc snippets
    % extracted from the output of `pandoc -s`
    \providecommand{\tightlist}{%
      \setlength{\itemsep}{0pt}\setlength{\parskip}{0pt}}
    \DefineVerbatimEnvironment{Highlighting}{Verbatim}{commandchars=\\\{\}}
    % Add ',fontsize=\small' for more characters per line
    \newenvironment{Shaded}{}{}
    \newcommand{\KeywordTok}[1]{\textcolor[rgb]{0.00,0.44,0.13}{\textbf{{#1}}}}
    \newcommand{\DataTypeTok}[1]{\textcolor[rgb]{0.56,0.13,0.00}{{#1}}}
    \newcommand{\DecValTok}[1]{\textcolor[rgb]{0.25,0.63,0.44}{{#1}}}
    \newcommand{\BaseNTok}[1]{\textcolor[rgb]{0.25,0.63,0.44}{{#1}}}
    \newcommand{\FloatTok}[1]{\textcolor[rgb]{0.25,0.63,0.44}{{#1}}}
    \newcommand{\CharTok}[1]{\textcolor[rgb]{0.25,0.44,0.63}{{#1}}}
    \newcommand{\StringTok}[1]{\textcolor[rgb]{0.25,0.44,0.63}{{#1}}}
    \newcommand{\CommentTok}[1]{\textcolor[rgb]{0.38,0.63,0.69}{\textit{{#1}}}}
    \newcommand{\OtherTok}[1]{\textcolor[rgb]{0.00,0.44,0.13}{{#1}}}
    \newcommand{\AlertTok}[1]{\textcolor[rgb]{1.00,0.00,0.00}{\textbf{{#1}}}}
    \newcommand{\FunctionTok}[1]{\textcolor[rgb]{0.02,0.16,0.49}{{#1}}}
    \newcommand{\RegionMarkerTok}[1]{{#1}}
    \newcommand{\ErrorTok}[1]{\textcolor[rgb]{1.00,0.00,0.00}{\textbf{{#1}}}}
    \newcommand{\NormalTok}[1]{{#1}}
    
    % Additional commands for more recent versions of Pandoc
    \newcommand{\ConstantTok}[1]{\textcolor[rgb]{0.53,0.00,0.00}{{#1}}}
    \newcommand{\SpecialCharTok}[1]{\textcolor[rgb]{0.25,0.44,0.63}{{#1}}}
    \newcommand{\VerbatimStringTok}[1]{\textcolor[rgb]{0.25,0.44,0.63}{{#1}}}
    \newcommand{\SpecialStringTok}[1]{\textcolor[rgb]{0.73,0.40,0.53}{{#1}}}
    \newcommand{\ImportTok}[1]{{#1}}
    \newcommand{\DocumentationTok}[1]{\textcolor[rgb]{0.73,0.13,0.13}{\textit{{#1}}}}
    \newcommand{\AnnotationTok}[1]{\textcolor[rgb]{0.38,0.63,0.69}{\textbf{\textit{{#1}}}}}
    \newcommand{\CommentVarTok}[1]{\textcolor[rgb]{0.38,0.63,0.69}{\textbf{\textit{{#1}}}}}
    \newcommand{\VariableTok}[1]{\textcolor[rgb]{0.10,0.09,0.49}{{#1}}}
    \newcommand{\ControlFlowTok}[1]{\textcolor[rgb]{0.00,0.44,0.13}{\textbf{{#1}}}}
    \newcommand{\OperatorTok}[1]{\textcolor[rgb]{0.40,0.40,0.40}{{#1}}}
    \newcommand{\BuiltInTok}[1]{{#1}}
    \newcommand{\ExtensionTok}[1]{{#1}}
    \newcommand{\PreprocessorTok}[1]{\textcolor[rgb]{0.74,0.48,0.00}{{#1}}}
    \newcommand{\AttributeTok}[1]{\textcolor[rgb]{0.49,0.56,0.16}{{#1}}}
    \newcommand{\InformationTok}[1]{\textcolor[rgb]{0.38,0.63,0.69}{\textbf{\textit{{#1}}}}}
    \newcommand{\WarningTok}[1]{\textcolor[rgb]{0.38,0.63,0.69}{\textbf{\textit{{#1}}}}}
    
    
    % Define a nice break command that doesn't care if a line doesn't already
    % exist.
    \def\br{\hspace*{\fill} \\* }
    % Math Jax compatibility definitions
    \def\gt{>}
    \def\lt{<}
    \let\Oldtex\TeX
    \let\Oldlatex\LaTeX
    \renewcommand{\TeX}{\textrm{\Oldtex}}
    \renewcommand{\LaTeX}{\textrm{\Oldlatex}}
    % Document parameters
    % Document title
    \title{TP2\_ex2}
    
    
    
    
    
% Pygments definitions
\makeatletter
\def\PY@reset{\let\PY@it=\relax \let\PY@bf=\relax%
    \let\PY@ul=\relax \let\PY@tc=\relax%
    \let\PY@bc=\relax \let\PY@ff=\relax}
\def\PY@tok#1{\csname PY@tok@#1\endcsname}
\def\PY@toks#1+{\ifx\relax#1\empty\else%
    \PY@tok{#1}\expandafter\PY@toks\fi}
\def\PY@do#1{\PY@bc{\PY@tc{\PY@ul{%
    \PY@it{\PY@bf{\PY@ff{#1}}}}}}}
\def\PY#1#2{\PY@reset\PY@toks#1+\relax+\PY@do{#2}}

\@namedef{PY@tok@w}{\def\PY@tc##1{\textcolor[rgb]{0.73,0.73,0.73}{##1}}}
\@namedef{PY@tok@c}{\let\PY@it=\textit\def\PY@tc##1{\textcolor[rgb]{0.24,0.48,0.48}{##1}}}
\@namedef{PY@tok@cp}{\def\PY@tc##1{\textcolor[rgb]{0.61,0.40,0.00}{##1}}}
\@namedef{PY@tok@k}{\let\PY@bf=\textbf\def\PY@tc##1{\textcolor[rgb]{0.00,0.50,0.00}{##1}}}
\@namedef{PY@tok@kp}{\def\PY@tc##1{\textcolor[rgb]{0.00,0.50,0.00}{##1}}}
\@namedef{PY@tok@kt}{\def\PY@tc##1{\textcolor[rgb]{0.69,0.00,0.25}{##1}}}
\@namedef{PY@tok@o}{\def\PY@tc##1{\textcolor[rgb]{0.40,0.40,0.40}{##1}}}
\@namedef{PY@tok@ow}{\let\PY@bf=\textbf\def\PY@tc##1{\textcolor[rgb]{0.67,0.13,1.00}{##1}}}
\@namedef{PY@tok@nb}{\def\PY@tc##1{\textcolor[rgb]{0.00,0.50,0.00}{##1}}}
\@namedef{PY@tok@nf}{\def\PY@tc##1{\textcolor[rgb]{0.00,0.00,1.00}{##1}}}
\@namedef{PY@tok@nc}{\let\PY@bf=\textbf\def\PY@tc##1{\textcolor[rgb]{0.00,0.00,1.00}{##1}}}
\@namedef{PY@tok@nn}{\let\PY@bf=\textbf\def\PY@tc##1{\textcolor[rgb]{0.00,0.00,1.00}{##1}}}
\@namedef{PY@tok@ne}{\let\PY@bf=\textbf\def\PY@tc##1{\textcolor[rgb]{0.80,0.25,0.22}{##1}}}
\@namedef{PY@tok@nv}{\def\PY@tc##1{\textcolor[rgb]{0.10,0.09,0.49}{##1}}}
\@namedef{PY@tok@no}{\def\PY@tc##1{\textcolor[rgb]{0.53,0.00,0.00}{##1}}}
\@namedef{PY@tok@nl}{\def\PY@tc##1{\textcolor[rgb]{0.46,0.46,0.00}{##1}}}
\@namedef{PY@tok@ni}{\let\PY@bf=\textbf\def\PY@tc##1{\textcolor[rgb]{0.44,0.44,0.44}{##1}}}
\@namedef{PY@tok@na}{\def\PY@tc##1{\textcolor[rgb]{0.41,0.47,0.13}{##1}}}
\@namedef{PY@tok@nt}{\let\PY@bf=\textbf\def\PY@tc##1{\textcolor[rgb]{0.00,0.50,0.00}{##1}}}
\@namedef{PY@tok@nd}{\def\PY@tc##1{\textcolor[rgb]{0.67,0.13,1.00}{##1}}}
\@namedef{PY@tok@s}{\def\PY@tc##1{\textcolor[rgb]{0.73,0.13,0.13}{##1}}}
\@namedef{PY@tok@sd}{\let\PY@it=\textit\def\PY@tc##1{\textcolor[rgb]{0.73,0.13,0.13}{##1}}}
\@namedef{PY@tok@si}{\let\PY@bf=\textbf\def\PY@tc##1{\textcolor[rgb]{0.64,0.35,0.47}{##1}}}
\@namedef{PY@tok@se}{\let\PY@bf=\textbf\def\PY@tc##1{\textcolor[rgb]{0.67,0.36,0.12}{##1}}}
\@namedef{PY@tok@sr}{\def\PY@tc##1{\textcolor[rgb]{0.64,0.35,0.47}{##1}}}
\@namedef{PY@tok@ss}{\def\PY@tc##1{\textcolor[rgb]{0.10,0.09,0.49}{##1}}}
\@namedef{PY@tok@sx}{\def\PY@tc##1{\textcolor[rgb]{0.00,0.50,0.00}{##1}}}
\@namedef{PY@tok@m}{\def\PY@tc##1{\textcolor[rgb]{0.40,0.40,0.40}{##1}}}
\@namedef{PY@tok@gh}{\let\PY@bf=\textbf\def\PY@tc##1{\textcolor[rgb]{0.00,0.00,0.50}{##1}}}
\@namedef{PY@tok@gu}{\let\PY@bf=\textbf\def\PY@tc##1{\textcolor[rgb]{0.50,0.00,0.50}{##1}}}
\@namedef{PY@tok@gd}{\def\PY@tc##1{\textcolor[rgb]{0.63,0.00,0.00}{##1}}}
\@namedef{PY@tok@gi}{\def\PY@tc##1{\textcolor[rgb]{0.00,0.52,0.00}{##1}}}
\@namedef{PY@tok@gr}{\def\PY@tc##1{\textcolor[rgb]{0.89,0.00,0.00}{##1}}}
\@namedef{PY@tok@ge}{\let\PY@it=\textit}
\@namedef{PY@tok@gs}{\let\PY@bf=\textbf}
\@namedef{PY@tok@gp}{\let\PY@bf=\textbf\def\PY@tc##1{\textcolor[rgb]{0.00,0.00,0.50}{##1}}}
\@namedef{PY@tok@go}{\def\PY@tc##1{\textcolor[rgb]{0.44,0.44,0.44}{##1}}}
\@namedef{PY@tok@gt}{\def\PY@tc##1{\textcolor[rgb]{0.00,0.27,0.87}{##1}}}
\@namedef{PY@tok@err}{\def\PY@bc##1{{\setlength{\fboxsep}{\string -\fboxrule}\fcolorbox[rgb]{1.00,0.00,0.00}{1,1,1}{\strut ##1}}}}
\@namedef{PY@tok@kc}{\let\PY@bf=\textbf\def\PY@tc##1{\textcolor[rgb]{0.00,0.50,0.00}{##1}}}
\@namedef{PY@tok@kd}{\let\PY@bf=\textbf\def\PY@tc##1{\textcolor[rgb]{0.00,0.50,0.00}{##1}}}
\@namedef{PY@tok@kn}{\let\PY@bf=\textbf\def\PY@tc##1{\textcolor[rgb]{0.00,0.50,0.00}{##1}}}
\@namedef{PY@tok@kr}{\let\PY@bf=\textbf\def\PY@tc##1{\textcolor[rgb]{0.00,0.50,0.00}{##1}}}
\@namedef{PY@tok@bp}{\def\PY@tc##1{\textcolor[rgb]{0.00,0.50,0.00}{##1}}}
\@namedef{PY@tok@fm}{\def\PY@tc##1{\textcolor[rgb]{0.00,0.00,1.00}{##1}}}
\@namedef{PY@tok@vc}{\def\PY@tc##1{\textcolor[rgb]{0.10,0.09,0.49}{##1}}}
\@namedef{PY@tok@vg}{\def\PY@tc##1{\textcolor[rgb]{0.10,0.09,0.49}{##1}}}
\@namedef{PY@tok@vi}{\def\PY@tc##1{\textcolor[rgb]{0.10,0.09,0.49}{##1}}}
\@namedef{PY@tok@vm}{\def\PY@tc##1{\textcolor[rgb]{0.10,0.09,0.49}{##1}}}
\@namedef{PY@tok@sa}{\def\PY@tc##1{\textcolor[rgb]{0.73,0.13,0.13}{##1}}}
\@namedef{PY@tok@sb}{\def\PY@tc##1{\textcolor[rgb]{0.73,0.13,0.13}{##1}}}
\@namedef{PY@tok@sc}{\def\PY@tc##1{\textcolor[rgb]{0.73,0.13,0.13}{##1}}}
\@namedef{PY@tok@dl}{\def\PY@tc##1{\textcolor[rgb]{0.73,0.13,0.13}{##1}}}
\@namedef{PY@tok@s2}{\def\PY@tc##1{\textcolor[rgb]{0.73,0.13,0.13}{##1}}}
\@namedef{PY@tok@sh}{\def\PY@tc##1{\textcolor[rgb]{0.73,0.13,0.13}{##1}}}
\@namedef{PY@tok@s1}{\def\PY@tc##1{\textcolor[rgb]{0.73,0.13,0.13}{##1}}}
\@namedef{PY@tok@mb}{\def\PY@tc##1{\textcolor[rgb]{0.40,0.40,0.40}{##1}}}
\@namedef{PY@tok@mf}{\def\PY@tc##1{\textcolor[rgb]{0.40,0.40,0.40}{##1}}}
\@namedef{PY@tok@mh}{\def\PY@tc##1{\textcolor[rgb]{0.40,0.40,0.40}{##1}}}
\@namedef{PY@tok@mi}{\def\PY@tc##1{\textcolor[rgb]{0.40,0.40,0.40}{##1}}}
\@namedef{PY@tok@il}{\def\PY@tc##1{\textcolor[rgb]{0.40,0.40,0.40}{##1}}}
\@namedef{PY@tok@mo}{\def\PY@tc##1{\textcolor[rgb]{0.40,0.40,0.40}{##1}}}
\@namedef{PY@tok@ch}{\let\PY@it=\textit\def\PY@tc##1{\textcolor[rgb]{0.24,0.48,0.48}{##1}}}
\@namedef{PY@tok@cm}{\let\PY@it=\textit\def\PY@tc##1{\textcolor[rgb]{0.24,0.48,0.48}{##1}}}
\@namedef{PY@tok@cpf}{\let\PY@it=\textit\def\PY@tc##1{\textcolor[rgb]{0.24,0.48,0.48}{##1}}}
\@namedef{PY@tok@c1}{\let\PY@it=\textit\def\PY@tc##1{\textcolor[rgb]{0.24,0.48,0.48}{##1}}}
\@namedef{PY@tok@cs}{\let\PY@it=\textit\def\PY@tc##1{\textcolor[rgb]{0.24,0.48,0.48}{##1}}}

\def\PYZbs{\char`\\}
\def\PYZus{\char`\_}
\def\PYZob{\char`\{}
\def\PYZcb{\char`\}}
\def\PYZca{\char`\^}
\def\PYZam{\char`\&}
\def\PYZlt{\char`\<}
\def\PYZgt{\char`\>}
\def\PYZsh{\char`\#}
\def\PYZpc{\char`\%}
\def\PYZdl{\char`\$}
\def\PYZhy{\char`\-}
\def\PYZsq{\char`\'}
\def\PYZdq{\char`\"}
\def\PYZti{\char`\~}
% for compatibility with earlier versions
\def\PYZat{@}
\def\PYZlb{[}
\def\PYZrb{]}
\makeatother


    % For linebreaks inside Verbatim environment from package fancyvrb. 
    \makeatletter
        \newbox\Wrappedcontinuationbox 
        \newbox\Wrappedvisiblespacebox 
        \newcommand*\Wrappedvisiblespace {\textcolor{red}{\textvisiblespace}} 
        \newcommand*\Wrappedcontinuationsymbol {\textcolor{red}{\llap{\tiny$\m@th\hookrightarrow$}}} 
        \newcommand*\Wrappedcontinuationindent {3ex } 
        \newcommand*\Wrappedafterbreak {\kern\Wrappedcontinuationindent\copy\Wrappedcontinuationbox} 
        % Take advantage of the already applied Pygments mark-up to insert 
        % potential linebreaks for TeX processing. 
        %        {, <, #, %, $, ' and ": go to next line. 
        %        _, }, ^, &, >, - and ~: stay at end of broken line. 
        % Use of \textquotesingle for straight quote. 
        \newcommand*\Wrappedbreaksatspecials {% 
            \def\PYGZus{\discretionary{\char`\_}{\Wrappedafterbreak}{\char`\_}}% 
            \def\PYGZob{\discretionary{}{\Wrappedafterbreak\char`\{}{\char`\{}}% 
            \def\PYGZcb{\discretionary{\char`\}}{\Wrappedafterbreak}{\char`\}}}% 
            \def\PYGZca{\discretionary{\char`\^}{\Wrappedafterbreak}{\char`\^}}% 
            \def\PYGZam{\discretionary{\char`\&}{\Wrappedafterbreak}{\char`\&}}% 
            \def\PYGZlt{\discretionary{}{\Wrappedafterbreak\char`\<}{\char`\<}}% 
            \def\PYGZgt{\discretionary{\char`\>}{\Wrappedafterbreak}{\char`\>}}% 
            \def\PYGZsh{\discretionary{}{\Wrappedafterbreak\char`\#}{\char`\#}}% 
            \def\PYGZpc{\discretionary{}{\Wrappedafterbreak\char`\%}{\char`\%}}% 
            \def\PYGZdl{\discretionary{}{\Wrappedafterbreak\char`\$}{\char`\$}}% 
            \def\PYGZhy{\discretionary{\char`\-}{\Wrappedafterbreak}{\char`\-}}% 
            \def\PYGZsq{\discretionary{}{\Wrappedafterbreak\textquotesingle}{\textquotesingle}}% 
            \def\PYGZdq{\discretionary{}{\Wrappedafterbreak\char`\"}{\char`\"}}% 
            \def\PYGZti{\discretionary{\char`\~}{\Wrappedafterbreak}{\char`\~}}% 
        } 
        % Some characters . , ; ? ! / are not pygmentized. 
        % This macro makes them "active" and they will insert potential linebreaks 
        \newcommand*\Wrappedbreaksatpunct {% 
            \lccode`\~`\.\lowercase{\def~}{\discretionary{\hbox{\char`\.}}{\Wrappedafterbreak}{\hbox{\char`\.}}}% 
            \lccode`\~`\,\lowercase{\def~}{\discretionary{\hbox{\char`\,}}{\Wrappedafterbreak}{\hbox{\char`\,}}}% 
            \lccode`\~`\;\lowercase{\def~}{\discretionary{\hbox{\char`\;}}{\Wrappedafterbreak}{\hbox{\char`\;}}}% 
            \lccode`\~`\:\lowercase{\def~}{\discretionary{\hbox{\char`\:}}{\Wrappedafterbreak}{\hbox{\char`\:}}}% 
            \lccode`\~`\?\lowercase{\def~}{\discretionary{\hbox{\char`\?}}{\Wrappedafterbreak}{\hbox{\char`\?}}}% 
            \lccode`\~`\!\lowercase{\def~}{\discretionary{\hbox{\char`\!}}{\Wrappedafterbreak}{\hbox{\char`\!}}}% 
            \lccode`\~`\/\lowercase{\def~}{\discretionary{\hbox{\char`\/}}{\Wrappedafterbreak}{\hbox{\char`\/}}}% 
            \catcode`\.\active
            \catcode`\,\active 
            \catcode`\;\active
            \catcode`\:\active
            \catcode`\?\active
            \catcode`\!\active
            \catcode`\/\active 
            \lccode`\~`\~ 	
        }
    \makeatother

    \let\OriginalVerbatim=\Verbatim
    \makeatletter
    \renewcommand{\Verbatim}[1][1]{%
        %\parskip\z@skip
        \sbox\Wrappedcontinuationbox {\Wrappedcontinuationsymbol}%
        \sbox\Wrappedvisiblespacebox {\FV@SetupFont\Wrappedvisiblespace}%
        \def\FancyVerbFormatLine ##1{\hsize\linewidth
            \vtop{\raggedright\hyphenpenalty\z@\exhyphenpenalty\z@
                \doublehyphendemerits\z@\finalhyphendemerits\z@
                \strut ##1\strut}%
        }%
        % If the linebreak is at a space, the latter will be displayed as visible
        % space at end of first line, and a continuation symbol starts next line.
        % Stretch/shrink are however usually zero for typewriter font.
        \def\FV@Space {%
            \nobreak\hskip\z@ plus\fontdimen3\font minus\fontdimen4\font
            \discretionary{\copy\Wrappedvisiblespacebox}{\Wrappedafterbreak}
            {\kern\fontdimen2\font}%
        }%
        
        % Allow breaks at special characters using \PYG... macros.
        \Wrappedbreaksatspecials
        % Breaks at punctuation characters . , ; ? ! and / need catcode=\active 	
        \OriginalVerbatim[#1,codes*=\Wrappedbreaksatpunct]%
    }
    \makeatother

    % Exact colors from NB
    \definecolor{incolor}{HTML}{303F9F}
    \definecolor{outcolor}{HTML}{D84315}
    \definecolor{cellborder}{HTML}{CFCFCF}
    \definecolor{cellbackground}{HTML}{F7F7F7}
    
    % prompt
    \makeatletter
    \newcommand{\boxspacing}{\kern\kvtcb@left@rule\kern\kvtcb@boxsep}
    \makeatother
    \newcommand{\prompt}[4]{
        {\ttfamily\llap{{\color{#2}[#3]:\hspace{3pt}#4}}\vspace{-\baselineskip}}
    }
    

    
    % Prevent overflowing lines due to hard-to-break entities
    \sloppy 
    % Setup hyperref package
    \hypersetup{
      breaklinks=true,  % so long urls are correctly broken across lines
      colorlinks=true,
      urlcolor=urlcolor,
      linkcolor=linkcolor,
      citecolor=citecolor,
      }
    % Slightly bigger margins than the latex defaults
    
    \geometry{verbose,tmargin=1in,bmargin=1in,lmargin=1in,rmargin=1in}
    
    

\begin{document}
    
    \maketitle
    
    

    
    \hypertarget{implementauxe7uxe3o}{%
\textbf{\large Implementação}\label{implementauxe7uxe3o}}

Começamos por importar alguns métodos úteis para a resolução do problema

O pygame será utilizado para uma visualização gráfica do problema
enunciado Para a utilização do mesmo é necessário instalá-lo (é uma
biblioteca externa):

\textbf{pip install pygame}

    \begin{tcolorbox}[breakable, size=fbox, boxrule=1pt, pad at break*=1mm,colback=cellbackground, colframe=cellborder]
\prompt{In}{incolor}{ }{\boxspacing}
\begin{Verbatim}[commandchars=\\\{\}]
\PY{k+kn}{from} \PY{n+nn}{random} \PY{k+kn}{import} \PY{n}{choices}
\PY{k+kn}{from} \PY{n+nn}{math} \PY{k+kn}{import} \PY{n}{floor}
\PY{k+kn}{from} \PY{n+nn}{copy} \PY{k+kn}{import} \PY{n}{deepcopy}
\PY{k+kn}{import} \PY{n+nn}{pygame}
\end{Verbatim}
\end{tcolorbox}

    \hypertarget{paruxe2metros-do-programa}{%
    \textbf{\large Parâmetros do programa}\label{paruxe2metros-do-programa}}

\begin{itemize}
\tightlist
\item
  N: corresponde ao tamanho da grelha de células normais
\item
  p: probabilidade de cada célula ``nascer'' viva
\item
  c: posição onde o centro com 3 x 3 células vivas iniciais se localiza
\end{itemize}

    \begin{tcolorbox}[breakable, size=fbox, boxrule=1pt, pad at break*=1mm,colback=cellbackground, colframe=cellborder]
\prompt{In}{incolor}{ }{\boxspacing}
\begin{Verbatim}[commandchars=\\\{\}]
\PY{n}{N} \PY{o}{=} \PY{l+m+mi}{15}
\PY{n}{p} \PY{o}{=} \PY{l+m+mi}{1}
\PY{n}{c} \PY{o}{=} \PY{p}{(}\PY{l+m+mi}{5}\PY{p}{,}\PY{l+m+mi}{5}\PY{p}{)}
\end{Verbatim}
\end{tcolorbox}

    \hypertarget{solver}{%
    \textbf{\large ``Solver''}\label{solver}}

Após analisar o problema, resolvemos não utilizar um solver que foi
trabalhado nas aulas. Fizemos o nosso próprio ``solver'' de forma a
modelar o problema de forma mais compreensiva e possívelmente mais
eficiente.

\hypertarget{funuxe7uxe3o-auxiliar-1---vizinhos_possiveis}{%
\textbf{\large Função auxiliar 1 -
vizinhos\_possiveis}\label{funuxe7uxe3o-auxiliar-1---vizinhos_possiveis}}

É uma função que recebe como parâmetros os valores das posições de x e y
e o tamanho máximo da matriz (N x N) que queremos considerar. A função
devolve todos os pares de posições x, y correspondente aos vizinhos dos
valores recebidos como parâmetro, de forma a filtrar os casos que não
fazem sentido (que saem da matriz)

    \begin{tcolorbox}[breakable, size=fbox, boxrule=1pt, pad at break*=1mm,colback=cellbackground, colframe=cellborder]
\prompt{In}{incolor}{ }{\boxspacing}
\begin{Verbatim}[commandchars=\\\{\}]
\PY{k}{def} \PY{n+nf}{vizinhos\PYZus{}possiveis}\PY{p}{(}\PY{n}{x}\PY{p}{,}\PY{n}{y}\PY{p}{,} \PY{n}{N}\PY{p}{)}\PY{p}{:}
    \PY{n}{v} \PY{o}{=}  \PY{p}{[}\PY{p}{(}\PY{n}{x}\PY{o}{+}\PY{l+m+mi}{1}\PY{p}{,}\PY{n}{y}\PY{p}{)}\PY{p}{,} \PY{p}{(}\PY{n}{x}\PY{o}{\PYZhy{}}\PY{l+m+mi}{1}\PY{p}{,}\PY{n}{y}\PY{p}{)}\PY{p}{,} \PY{p}{(}\PY{n}{x}\PY{p}{,}\PY{n}{y}\PY{o}{\PYZhy{}}\PY{l+m+mi}{1}\PY{p}{)}\PY{p}{,} \PY{p}{(}\PY{n}{x}\PY{p}{,}\PY{n}{y}\PY{o}{+}\PY{l+m+mi}{1}\PY{p}{)}\PY{p}{,} \PY{p}{(}\PY{n}{x}\PY{o}{\PYZhy{}}\PY{l+m+mi}{1}\PY{p}{,} \PY{n}{y}\PY{o}{+}\PY{l+m+mi}{1}\PY{p}{)}\PY{p}{,} \PY{p}{(}\PY{n}{x}\PY{o}{\PYZhy{}}\PY{l+m+mi}{1}\PY{p}{,}\PY{n}{y}\PY{o}{\PYZhy{}}\PY{l+m+mi}{1}\PY{p}{)}\PY{p}{,} \PY{p}{(}\PY{n}{x}\PY{o}{+}\PY{l+m+mi}{1}\PY{p}{,} \PY{n}{y}\PY{o}{+}\PY{l+m+mi}{1}\PY{p}{)}\PY{p}{,} \PY{p}{(}\PY{n}{x}\PY{o}{+}\PY{l+m+mi}{1}\PY{p}{,}\PY{n}{y}\PY{o}{\PYZhy{}}\PY{l+m+mi}{1}\PY{p}{)}\PY{p}{]}
    \PY{k}{return} \PY{n+nb}{list}\PY{p}{(}\PY{n+nb}{filter}\PY{p}{(}\PY{k}{lambda} \PY{n}{t}\PY{p}{:} \PY{n}{t}\PY{p}{[}\PY{l+m+mi}{0}\PY{p}{]} \PY{o}{\PYZgt{}}\PY{o}{=} \PY{l+m+mi}{0} \PY{o+ow}{and} \PY{n}{t}\PY{p}{[}\PY{l+m+mi}{0}\PY{p}{]} \PY{o}{\PYZlt{}} \PY{n}{N} \PY{o+ow}{and} \PY{n}{t}\PY{p}{[}\PY{l+m+mi}{1}\PY{p}{]} \PY{o}{\PYZgt{}}\PY{o}{=} \PY{l+m+mi}{0} \PY{o+ow}{and} \PY{n}{t}\PY{p}{[}\PY{l+m+mi}{1}\PY{p}{]} \PY{o}{\PYZlt{}} \PY{n}{N}\PY{p}{,} \PY{n}{v}\PY{p}{)}\PY{p}{)}
\end{Verbatim}
\end{tcolorbox}

    \hypertarget{funuxe7uxe3o-auxiliar-2---celulas_vizinhas_vivas}{%
    \textbf{\large Função auxiliar 2 -
celulas\_vizinhas\_vivas}\label{funuxe7uxe3o-auxiliar-2---celulas_vizinhas_vivas}}

É uma função que recebe como parâmetro uma matriz de células e uma lista
de vizinhos, isto é, uma lista de pares de posições x e y vizinhos a uma
determinada célula. A função devolve a quantidade de células vivas
dentre os vizinhos.

    \begin{tcolorbox}[breakable, size=fbox, boxrule=1pt, pad at break*=1mm,colback=cellbackground, colframe=cellborder]
\prompt{In}{incolor}{ }{\boxspacing}
\begin{Verbatim}[commandchars=\\\{\}]
\PY{k}{def} \PY{n+nf}{celulas\PYZus{}vizinhas\PYZus{}vivas}\PY{p}{(}\PY{n}{mapa}\PY{p}{,} \PY{n}{vizinhos}\PY{p}{)}\PY{p}{:}
    \PY{n}{vivas} \PY{o}{=} \PY{l+m+mi}{0}
    \PY{k}{for} \PY{n}{x}\PY{p}{,}\PY{n}{y} \PY{o+ow}{in} \PY{n}{vizinhos}\PY{p}{:}
        \PY{k}{if} \PY{n}{mapa}\PY{p}{[}\PY{n}{y}\PY{p}{]}\PY{p}{[}\PY{n}{x}\PY{p}{]}\PY{p}{:}
            \PY{n}{vivas} \PY{o}{+}\PY{o}{=} \PY{l+m+mi}{1}
    \PY{k}{return} \PY{n}{vivas}
\end{Verbatim}
\end{tcolorbox}

    \hypertarget{funuxe7uxe3o-auxiliar-3---list_to_key}{%
    \textbf{\large Função auxiliar 3 -
list\_to\_key}\label{funuxe7uxe3o-auxiliar-3---list_to_key}}

É uma função que recebe como parâmetro uma lista de listas e devolve um
tuplo de tuplos com os mesmos elementos. O ponto todo desta função é que
as listas são estruturas mutáveis, enquanto que os tuplos são imutáveis.
A utilização desta função vai ficar mais clara mais a frente

    \begin{tcolorbox}[breakable, size=fbox, boxrule=1pt, pad at break*=1mm,colback=cellbackground, colframe=cellborder]
\prompt{In}{incolor}{ }{\boxspacing}
\begin{Verbatim}[commandchars=\\\{\}]
\PY{k}{def} \PY{n+nf}{list\PYZus{}to\PYZus{}key}\PY{p}{(}\PY{n}{lista}\PY{p}{)}\PY{p}{:}
    \PY{k}{return} \PY{n+nb}{tuple}\PY{p}{(}\PY{n+nb}{tuple}\PY{p}{(}\PY{n}{l}\PY{p}{)} \PY{k}{for} \PY{n}{l} \PY{o+ow}{in} \PY{n}{lista}\PY{p}{)}
\end{Verbatim}
\end{tcolorbox}

    \hypertarget{funuxe7uxe3o-estado_inicial}{%
    \textbf{\large Função estado\_inicial}\label{funuxe7uxe3o-estado_inicial}}

É uma função que recebe como parâmetro uma matriz de células, um par que
corresponde a posição do centro de células vivas 3 x 3, a probabilidade
das células de borda ``nascerem'' vivas e o tamanho da matriz
correspondente as células normais. Esta função não retorna nada, mas
altera diretamente a matriz recebida como parâmetro.

    \begin{tcolorbox}[breakable, size=fbox, boxrule=1pt, pad at break*=1mm,colback=cellbackground, colframe=cellborder]
\prompt{In}{incolor}{ }{\boxspacing}
\begin{Verbatim}[commandchars=\\\{\}]
\PY{k}{def} \PY{n+nf}{estado\PYZus{}inicial}\PY{p}{(}\PY{n}{lista\PYZus{}aux}\PY{p}{,} \PY{n}{centro}\PY{p}{,} \PY{n}{probabilidade}\PY{p}{,} \PY{n}{N}\PY{p}{)}\PY{p}{:}
    \PY{n}{lista\PYZus{}aux}\PY{p}{[}\PY{n}{centro}\PY{p}{[}\PY{l+m+mi}{1}\PY{p}{]}\PY{p}{]}\PY{p}{[}\PY{n}{centro}\PY{p}{[}\PY{l+m+mi}{0}\PY{p}{]}\PY{p}{]} \PY{o}{=} \PY{l+m+mi}{1}

    \PY{k}{for} \PY{n}{x}\PY{p}{,}\PY{n}{y} \PY{o+ow}{in} \PY{n}{vizinhos\PYZus{}possiveis}\PY{p}{(}\PY{n}{centro}\PY{p}{[}\PY{l+m+mi}{0}\PY{p}{]}\PY{p}{,}\PY{n}{centro}\PY{p}{[}\PY{l+m+mi}{1}\PY{p}{]}\PY{p}{,} \PY{n}{N}\PY{o}{+}\PY{l+m+mi}{1}\PY{p}{)}\PY{p}{:}
        \PY{n}{lista\PYZus{}aux}\PY{p}{[}\PY{n}{y}\PY{p}{]}\PY{p}{[}\PY{n}{x}\PY{p}{]} \PY{o}{=} \PY{l+m+mi}{1}
    
    \PY{n}{lista\PYZus{}aux}\PY{p}{[}\PY{l+m+mi}{0}\PY{p}{]}\PY{p}{[}\PY{l+m+mi}{0}\PY{p}{]} \PY{o}{=} \PY{n}{choices}\PY{p}{(}\PY{p}{[}\PY{l+m+mi}{1}\PY{p}{,} \PY{l+m+mi}{0}\PY{p}{]}\PY{p}{,} \PY{p}{[}\PY{n}{probabilidade}\PY{p}{,} \PY{l+m+mi}{1}\PY{o}{\PYZhy{}}\PY{n}{probabilidade}\PY{p}{]}\PY{p}{)}\PY{p}{[}\PY{l+m+mi}{0}\PY{p}{]}

    \PY{k}{for} \PY{n}{i} \PY{o+ow}{in} \PY{n+nb}{range}\PY{p}{(}\PY{l+m+mi}{1}\PY{p}{,} \PY{n}{N} \PY{o}{+} \PY{l+m+mi}{1}\PY{p}{)}\PY{p}{:}
        \PY{n}{lista\PYZus{}aux}\PY{p}{[}\PY{n}{i}\PY{p}{]}\PY{p}{[}\PY{l+m+mi}{0}\PY{p}{]} \PY{o}{=} \PY{n}{choices}\PY{p}{(}\PY{p}{[}\PY{l+m+mi}{1}\PY{p}{,} \PY{l+m+mi}{0}\PY{p}{]}\PY{p}{,} \PY{p}{[}\PY{n}{probabilidade}\PY{p}{,} \PY{l+m+mi}{1}\PY{o}{\PYZhy{}}\PY{n}{probabilidade}\PY{p}{]}\PY{p}{)}\PY{p}{[}\PY{l+m+mi}{0}\PY{p}{]}
        \PY{n}{lista\PYZus{}aux}\PY{p}{[}\PY{l+m+mi}{0}\PY{p}{]}\PY{p}{[}\PY{n}{i}\PY{p}{]} \PY{o}{=} \PY{n}{choices}\PY{p}{(}\PY{p}{[}\PY{l+m+mi}{1}\PY{p}{,} \PY{l+m+mi}{0}\PY{p}{]}\PY{p}{,} \PY{p}{[}\PY{n}{probabilidade}\PY{p}{,} \PY{l+m+mi}{1}\PY{o}{\PYZhy{}}\PY{n}{probabilidade}\PY{p}{]}\PY{p}{)}\PY{p}{[}\PY{l+m+mi}{0}\PY{p}{]}
\end{Verbatim}
\end{tcolorbox}

    \hypertarget{funuxe7uxe3o-trans_estado}{%
    \textbf{\large Função trans\_estado}\label{funuxe7uxe3o-trans_estado}}

É uma função que recebe como parâmetro a matriz de células corresponde
ao estado atual e o tamanho da parte correspondente as células normais.
A função devolve a matriz de células corresponde ao próximo estado.

    \begin{tcolorbox}[breakable, size=fbox, boxrule=1pt, pad at break*=1mm,colback=cellbackground, colframe=cellborder]
\prompt{In}{incolor}{ }{\boxspacing}
\begin{Verbatim}[commandchars=\\\{\}]
\PY{k}{def} \PY{n+nf}{trans\PYZus{}estado}\PY{p}{(}\PY{n}{atual\PYZus{}estado}\PY{p}{,} \PY{n}{N}\PY{p}{)}\PY{p}{:}
    \PY{n}{prox\PYZus{}estado} \PY{o}{=} \PY{n}{deepcopy}\PY{p}{(}\PY{n}{atual\PYZus{}estado}\PY{p}{)}
    \PY{k}{for} \PY{n}{i} \PY{o+ow}{in} \PY{n+nb}{range}\PY{p}{(}\PY{l+m+mi}{1}\PY{p}{,} \PY{n}{N}\PY{o}{+}\PY{l+m+mi}{1}\PY{p}{)}\PY{p}{:}
        \PY{k}{for} \PY{n}{j} \PY{o+ow}{in} \PY{n+nb}{range}\PY{p}{(}\PY{l+m+mi}{1}\PY{p}{,} \PY{n}{N}\PY{o}{+}\PY{l+m+mi}{1}\PY{p}{)}\PY{p}{:}
            \PY{n}{vizinhos} \PY{o}{=} \PY{n}{vizinhos\PYZus{}possiveis}\PY{p}{(}\PY{n}{i}\PY{p}{,}\PY{n}{j}\PY{p}{,}\PY{n}{N}\PY{p}{)}
            \PY{n}{vivas} \PY{o}{=} \PY{n}{celulas\PYZus{}vizinhas\PYZus{}vivas}\PY{p}{(}\PY{n}{atual\PYZus{}estado}\PY{p}{,} \PY{n}{vizinhos}\PY{p}{)}
            \PY{k}{if} \PY{o+ow}{not} \PY{n}{atual\PYZus{}estado}\PY{p}{[}\PY{n}{i}\PY{p}{]}\PY{p}{[}\PY{n}{j}\PY{p}{]} \PY{o+ow}{and} \PY{n}{vivas} \PY{o}{==} \PY{l+m+mi}{3}\PY{p}{:}
                \PY{n}{prox\PYZus{}estado}\PY{p}{[}\PY{n}{i}\PY{p}{]}\PY{p}{[}\PY{n}{j}\PY{p}{]} \PY{o}{=} \PY{l+m+mi}{1}
            \PY{k}{elif} \PY{n}{vivas} \PY{o}{!=} \PY{l+m+mi}{2} \PY{o+ow}{and} \PY{n}{vivas} \PY{o}{!=} \PY{l+m+mi}{3}\PY{p}{:}
                \PY{n}{prox\PYZus{}estado}\PY{p}{[}\PY{n}{i}\PY{p}{]}\PY{p}{[}\PY{n}{j}\PY{p}{]} \PY{o}{=} \PY{l+m+mi}{0}

    \PY{k}{return} \PY{n}{prox\PYZus{}estado}
\end{Verbatim}
\end{tcolorbox}

    \hypertarget{funuxe7uxe3o-verifica_i}{%
    \textbf{\large Função verifica\_i}\label{funuxe7uxe3o-verifica_i}}

É uma função que recebe um dicionário dos estados acessíveis com
parâmetro. A função verifica se todos os estados acessíveis contém pelo
menos uma célula viva.

    \begin{tcolorbox}[breakable, size=fbox, boxrule=1pt, pad at break*=1mm,colback=cellbackground, colframe=cellborder]
\prompt{In}{incolor}{ }{\boxspacing}
\begin{Verbatim}[commandchars=\\\{\}]
\PY{k}{def} \PY{n+nf}{verifica\PYZus{}i}\PY{p}{(}\PY{n}{estados}\PY{p}{)}\PY{p}{:}
    \PY{n}{verifica} \PY{o}{=} \PY{k+kc}{True}
    \PY{k}{for} \PY{n}{estado} \PY{o+ow}{in} \PY{n}{estados}\PY{o}{.}\PY{n}{keys}\PY{p}{(}\PY{p}{)}\PY{p}{:}
        \PY{n}{aux} \PY{o}{=} \PY{p}{[}\PY{n}{elem} \PY{k}{for} \PY{n}{vetor} \PY{o+ow}{in} \PY{n}{estado} \PY{k}{for} \PY{n}{elem} \PY{o+ow}{in} \PY{n}{vetor}\PY{p}{]}
        \PY{k}{if} \PY{n+nb}{sum}\PY{p}{(}\PY{n}{aux}\PY{p}{)} \PY{o}{==} \PY{l+m+mi}{0}\PY{p}{:}
            \PY{n}{verifica} \PY{o}{=} \PY{k+kc}{False}
            \PY{k}{break}

    \PY{k}{return} \PY{n}{verifica}
\end{Verbatim}
\end{tcolorbox}

    \hypertarget{funuxe7uxe3o-verifica_ii}{%
    \textbf{\large Função verifica\_ii}\label{funuxe7uxe3o-verifica_ii}}

É uma função que recebe um dicionário dos estados acessíveis como
parâmetro e o tamanho da matriz correspondente as células normais. A
função verifica que toda a célula normal está viva pelo menos uma vez em
algum estado acessível

    \begin{tcolorbox}[breakable, size=fbox, boxrule=1pt, pad at break*=1mm,colback=cellbackground, colframe=cellborder]
\prompt{In}{incolor}{ }{\boxspacing}
\begin{Verbatim}[commandchars=\\\{\}]
\PY{k}{def} \PY{n+nf}{verifica\PYZus{}ii}\PY{p}{(}\PY{n}{estados}\PY{p}{,} \PY{n}{N}\PY{p}{)}\PY{p}{:}
    \PY{n}{aux} \PY{o}{=} \PY{p}{[}\PY{p}{[}\PY{l+m+mi}{0} \PY{k}{for} \PY{n}{j} \PY{o+ow}{in} \PY{n+nb}{range}\PY{p}{(}\PY{n}{N}\PY{p}{)}\PY{p}{]} \PY{k}{for} \PY{n}{i} \PY{o+ow}{in} \PY{n+nb}{range}\PY{p}{(}\PY{n}{N}\PY{p}{)}\PY{p}{]}
    \PY{k}{for} \PY{n}{estado} \PY{o+ow}{in} \PY{n}{estados}\PY{o}{.}\PY{n}{keys}\PY{p}{(}\PY{p}{)}\PY{p}{:}
        \PY{k}{for} \PY{n}{i} \PY{o+ow}{in} \PY{n+nb}{range}\PY{p}{(}\PY{l+m+mi}{1}\PY{p}{,} \PY{n}{N}\PY{o}{+}\PY{l+m+mi}{1}\PY{p}{)}\PY{p}{:}
            \PY{k}{for} \PY{n}{j} \PY{o+ow}{in} \PY{n+nb}{range}\PY{p}{(}\PY{l+m+mi}{1}\PY{p}{,} \PY{n}{N}\PY{o}{+}\PY{l+m+mi}{1}\PY{p}{)}\PY{p}{:}
                \PY{n}{aux}\PY{p}{[}\PY{n}{i}\PY{o}{\PYZhy{}}\PY{l+m+mi}{1}\PY{p}{]}\PY{p}{[}\PY{n}{j}\PY{o}{\PYZhy{}}\PY{l+m+mi}{1}\PY{p}{]} \PY{o}{+}\PY{o}{=} \PY{n}{estado}\PY{p}{[}\PY{n}{i}\PY{p}{]}\PY{p}{[}\PY{n}{j}\PY{p}{]}

    \PY{n}{verifica} \PY{o}{=} \PY{k+kc}{True}
    \PY{n}{i} \PY{o}{=} \PY{l+m+mi}{0}
    \PY{k}{while} \PY{n}{verifica} \PY{o+ow}{and} \PY{n}{i} \PY{o}{\PYZlt{}} \PY{n}{N}\PY{p}{:}
        \PY{k}{for} \PY{n}{j} \PY{o+ow}{in} \PY{n+nb}{range}\PY{p}{(}\PY{n}{N}\PY{p}{)}\PY{p}{:}
            \PY{k}{if} \PY{n}{aux}\PY{p}{[}\PY{n}{i}\PY{p}{]}\PY{p}{[}\PY{n}{j}\PY{p}{]} \PY{o}{==} \PY{l+m+mi}{0}\PY{p}{:}
                \PY{n}{verifica} \PY{o}{=} \PY{k+kc}{False}
                \PY{k}{break}
        \PY{n}{i} \PY{o}{+}\PY{o}{=} \PY{l+m+mi}{1}
        
    \PY{k}{return} \PY{n}{verifica}
\end{Verbatim}
\end{tcolorbox}

    \hypertarget{criauxe7uxe3o-da-muxe1quina-de-estados-finita}{%
    \pagebreak
    \textbf{\large Criação da máquina de estados
finita}\label{criauxe7uxe3o-da-muxe1quina-de-estados-finita}}

Como nossa matriz é finita então nossa máquina também terá um número
finito de estados. Basta considerarmos que cada combinação diferente de
células vivas e mortas de uma matriz representa um estado único. Como
cada estado é único, podemos pensar em estados como chaves de um
dicionário, em que o valor associado a chave corresponde o seu estado
(valor inteiro). Uma chave tem que ser um estado imutável, por isto,
convertemos uma lista de listas em tuplo de tuplos sempre que queremos
trabalhar com as chaves. A transição de um mesmo estado para o outro é
sempre a mesma e como temos um número finito de estados, sabemos que
necessariamente encontraremos um ciclo e este será único, ou seja, a
partir daqui não surgirá estados diferentes daqueles que já temos e
portanto podemos terminar a construção da máquina. Para facilitar a
busca futura, invertemos as chaves com os valores, para aceder
diretamente a um estado (inteiro) específico.

    \begin{tcolorbox}[breakable, size=fbox, boxrule=1pt, pad at break*=1mm,colback=cellbackground, colframe=cellborder]
\prompt{In}{incolor}{ }{\boxspacing}
\begin{Verbatim}[commandchars=\\\{\}]
\PY{n}{estados} \PY{o}{=} \PY{p}{\PYZob{}}\PY{p}{\PYZcb{}}
\PY{n}{trans} \PY{o}{=} \PY{p}{\PYZob{}}\PY{p}{\PYZcb{}}

\PY{n}{lista\PYZus{}aux} \PY{o}{=} \PY{p}{[}\PY{p}{[}\PY{l+m+mi}{0} \PY{k}{for} \PY{n}{j} \PY{o+ow}{in} \PY{n+nb}{range}\PY{p}{(}\PY{n}{N}\PY{o}{+}\PY{l+m+mi}{1}\PY{p}{)}\PY{p}{]} \PY{k}{for} \PY{n}{i} \PY{o+ow}{in} \PY{n+nb}{range}\PY{p}{(}\PY{n}{N}\PY{o}{+}\PY{l+m+mi}{1}\PY{p}{)}\PY{p}{]}

\PY{n}{estado\PYZus{}inicial}\PY{p}{(}\PY{n}{lista\PYZus{}aux}\PY{p}{,} \PY{n}{c}\PY{p}{,} \PY{n}{p}\PY{p}{,} \PY{n}{N}\PY{p}{)}
\PY{n}{estados}\PY{p}{[}\PY{n}{list\PYZus{}to\PYZus{}key}\PY{p}{(}\PY{n}{lista\PYZus{}aux}\PY{p}{)}\PY{p}{]} \PY{o}{=} \PY{l+m+mi}{1}

\PY{n}{estado} \PY{o}{=} \PY{l+m+mi}{2}

\PY{n}{lista\PYZus{}aux} \PY{o}{=} \PY{n}{trans\PYZus{}estado}\PY{p}{(}\PY{n}{lista\PYZus{}aux}\PY{p}{,} \PY{n}{N}\PY{p}{)}
\PY{n}{chave} \PY{o}{=} \PY{n}{list\PYZus{}to\PYZus{}key}\PY{p}{(}\PY{n}{lista\PYZus{}aux}\PY{p}{)}
\PY{k}{while} \PY{n}{chave} \PY{o+ow}{not} \PY{o+ow}{in} \PY{n}{estados}\PY{p}{:}
    \PY{n}{estados}\PY{p}{[}\PY{n}{chave}\PY{p}{]} \PY{o}{=} \PY{n}{estado}
    \PY{n}{trans}\PY{p}{[}\PY{n}{estado}\PY{o}{\PYZhy{}}\PY{l+m+mi}{1}\PY{p}{]} \PY{o}{=} \PY{n}{estado}
    \PY{n}{estado} \PY{o}{+}\PY{o}{=} \PY{l+m+mi}{1}
    \PY{n}{lista\PYZus{}aux} \PY{o}{=} \PY{n}{trans\PYZus{}estado}\PY{p}{(}\PY{n}{lista\PYZus{}aux}\PY{p}{,} \PY{n}{N}\PY{p}{)}
    \PY{n}{chave} \PY{o}{=} \PY{n}{list\PYZus{}to\PYZus{}key}\PY{p}{(}\PY{n}{lista\PYZus{}aux}\PY{p}{)}

\PY{n}{trans}\PY{p}{[}\PY{n}{estado}\PY{o}{\PYZhy{}}\PY{l+m+mi}{1}\PY{p}{]} \PY{o}{=} \PY{n}{estados}\PY{p}{[}\PY{n}{chave}\PY{p}{]}

\PY{c+c1}{\PYZsh{} verificar se as propriedades i e ii são válidas}

\PY{n+nb}{print}\PY{p}{(}\PY{n}{verifica\PYZus{}i}\PY{p}{(}\PY{n}{estados}\PY{p}{)}\PY{p}{)}
\PY{n+nb}{print}\PY{p}{(}\PY{n}{verifica\PYZus{}ii}\PY{p}{(}\PY{n}{estados}\PY{p}{,} \PY{n}{N}\PY{p}{)}\PY{p}{)}

\PY{c+c1}{\PYZsh{} Inverte as chaves com os valores}

\PY{n}{estados} \PY{o}{=} \PY{p}{\PYZob{}}\PY{n}{v}\PY{p}{:} \PY{n}{k} \PY{k}{for} \PY{n}{k}\PY{p}{,} \PY{n}{v} \PY{o+ow}{in} \PY{n}{estados}\PY{o}{.}\PY{n}{items}\PY{p}{(}\PY{p}{)}\PY{p}{\PYZcb{}}
\end{Verbatim}
\end{tcolorbox}

    \hypertarget{pygame}{%
    \textbf{\Large pygame}\label{pygame}}

Não explicaremos aqui os conceitos relativos a esta biblioteca porque
foge ao tema do trabalho, mas apresentaremos o código que construímos
para a construção da parte que permite visualizar o problema de forma
gráfica. É preciso dar run all para funcionar corretamente (escolher
simulação manual ou automática, colocando àquela que não foi escolhida
em comentário)

    \begin{tcolorbox}[breakable, size=fbox, boxrule=1pt, pad at break*=1mm,colback=cellbackground, colframe=cellborder]
\prompt{In}{incolor}{ }{\boxspacing}
\begin{Verbatim}[commandchars=\\\{\}]
\PY{n}{pygame}\PY{o}{.}\PY{n}{init}\PY{p}{(}\PY{p}{)}
\end{Verbatim}
\end{tcolorbox}

    \hypertarget{estado-inicial}{%
    \textbf{\large Estado Inicial}\label{estado-inicial}}

    \begin{tcolorbox}[breakable, size=fbox, boxrule=1pt, pad at break*=1mm,colback=cellbackground, colframe=cellborder]
\prompt{In}{incolor}{ }{\boxspacing}
\begin{Verbatim}[commandchars=\\\{\}]
\PY{n}{MAX\PYZus{}SIMULATION\PYZus{}DIMENSION} \PY{o}{=} \PY{l+m+mi}{700}
\PY{n}{BLOCK\PYZus{}SIZE} \PY{o}{=} \PY{n}{floor}\PY{p}{(}\PY{n}{MAX\PYZus{}SIMULATION\PYZus{}DIMENSION}\PY{o}{/}\PY{p}{(}\PY{n}{N}\PY{o}{+}\PY{l+m+mi}{1}\PY{p}{)}\PY{p}{)}
\PY{n}{SIMULATION\PYZus{}DIMENSION} \PY{o}{=} \PY{n}{BLOCK\PYZus{}SIZE} \PY{o}{*} \PY{p}{(}\PY{n}{N}\PY{o}{+}\PY{l+m+mi}{1}\PY{p}{)}

\PY{k}{def} \PY{n+nf}{draw\PYZus{}cells}\PY{p}{(}\PY{n}{celulas}\PY{p}{,} \PY{n}{N}\PY{p}{)}\PY{p}{:}
    \PY{k}{for} \PY{n}{i} \PY{o+ow}{in} \PY{n+nb}{range}\PY{p}{(}\PY{l+m+mi}{0}\PY{p}{,} \PY{n}{N}\PY{o}{+}\PY{l+m+mi}{1}\PY{p}{)}\PY{p}{:}
        \PY{k}{for} \PY{n}{j} \PY{o+ow}{in} \PY{n+nb}{range}\PY{p}{(}\PY{l+m+mi}{0}\PY{p}{,} \PY{n}{N}\PY{o}{+}\PY{l+m+mi}{1}\PY{p}{)}\PY{p}{:}
            \PY{k}{if} \PY{n}{celulas}\PY{p}{[}\PY{n}{i}\PY{p}{]}\PY{p}{[}\PY{n}{j}\PY{p}{]}\PY{p}{:}
                \PY{n}{x} \PY{o}{=} \PY{n}{BLOCK\PYZus{}SIZE} \PY{o}{*} \PY{n}{i}
                \PY{n}{y} \PY{o}{=}  \PY{n}{BLOCK\PYZus{}SIZE} \PY{o}{*} \PY{n}{j} \PY{o}{+} \PY{l+m+mi}{100}
                \PY{n}{rect} \PY{o}{=} \PY{n}{pygame}\PY{o}{.}\PY{n}{draw}\PY{o}{.}\PY{n}{rect}\PY{p}{(}\PY{n}{screen}\PY{p}{,} \PY{p}{(}\PY{l+m+mi}{0}\PY{p}{,}\PY{l+m+mi}{0}\PY{p}{,}\PY{l+m+mi}{0}\PY{p}{)}\PY{p}{,} \PY{p}{[}\PY{n}{x}\PY{p}{,} \PY{n}{y}\PY{p}{,} \PY{n}{BLOCK\PYZus{}SIZE}\PY{p}{,} \PY{n}{BLOCK\PYZus{}SIZE}\PY{p}{]}\PY{p}{)}
                \PY{n}{pygame}\PY{o}{.}\PY{n}{draw}\PY{o}{.}\PY{n}{rect}\PY{p}{(}\PY{n}{screen}\PY{p}{,} \PY{p}{(}\PY{l+m+mi}{255}\PY{p}{,}\PY{l+m+mi}{0}\PY{p}{,}\PY{l+m+mi}{0}\PY{p}{)}\PY{p}{,} \PY{n}{rect}\PY{p}{,} \PY{l+m+mi}{1}\PY{p}{)}

\PY{n}{screen} \PY{o}{=} \PY{n}{pygame}\PY{o}{.}\PY{n}{display}\PY{o}{.}\PY{n}{set\PYZus{}mode}\PY{p}{(}\PY{p}{(}\PY{n}{SIMULATION\PYZus{}DIMENSION}\PY{p}{,} \PY{n}{SIMULATION\PYZus{}DIMENSION}\PY{o}{+}\PY{l+m+mi}{100}\PY{p}{)}\PY{p}{)}
\PY{n}{screen}\PY{o}{.}\PY{n}{fill}\PY{p}{(}\PY{p}{(}\PY{l+m+mi}{255}\PY{p}{,}\PY{l+m+mi}{255}\PY{p}{,}\PY{l+m+mi}{255}\PY{p}{)}\PY{p}{)}

\PY{n}{font} \PY{o}{=} \PY{n}{pygame}\PY{o}{.}\PY{n}{font}\PY{o}{.}\PY{n}{Font}\PY{p}{(}\PY{l+s+s1}{\PYZsq{}}\PY{l+s+s1}{freesansbold.ttf}\PY{l+s+s1}{\PYZsq{}}\PY{p}{,} \PY{l+m+mi}{32}\PY{p}{)}
\PY{n}{text} \PY{o}{=} \PY{n}{font}\PY{o}{.}\PY{n}{render}\PY{p}{(}\PY{l+s+s1}{\PYZsq{}}\PY{l+s+s1}{ESTADO 1}\PY{l+s+s1}{\PYZsq{}}\PY{p}{,} \PY{k+kc}{True}\PY{p}{,} \PY{p}{(}\PY{l+m+mi}{255}\PY{p}{,}\PY{l+m+mi}{0}\PY{p}{,}\PY{l+m+mi}{0}\PY{p}{)}\PY{p}{,} \PY{p}{(}\PY{l+m+mi}{255}\PY{p}{,}\PY{l+m+mi}{255}\PY{p}{,}\PY{l+m+mi}{255}\PY{p}{)}\PY{p}{)}
\PY{n}{textRect} \PY{o}{=} \PY{n}{text}\PY{o}{.}\PY{n}{get\PYZus{}rect}\PY{p}{(}\PY{p}{)}
\PY{n}{textRect}\PY{o}{.}\PY{n}{center} \PY{o}{=} \PY{p}{(}\PY{n}{SIMULATION\PYZus{}DIMENSION} \PY{o}{/}\PY{o}{/} \PY{l+m+mi}{2}\PY{p}{,} \PY{l+m+mi}{50}\PY{p}{)}

\PY{n}{estado} \PY{o}{=} \PY{l+m+mi}{1}
\PY{n}{draw\PYZus{}cells}\PY{p}{(}\PY{n}{estados}\PY{p}{[}\PY{n}{estado}\PY{p}{]}\PY{p}{,} \PY{n}{N}\PY{p}{)}
\PY{n}{pygame}\PY{o}{.}\PY{n}{display}\PY{o}{.}\PY{n}{update}\PY{p}{(}\PY{p}{)}
\end{Verbatim}
\end{tcolorbox}

    \hypertarget{simulauxe7uxe3o-manual-seta-esquerda-e-direita}{%
    \textbf{\large Simulação Manual (Seta esquerda e
direita)}\label{simulauxe7uxe3o-manual-seta-esquerda-e-direita}}

    \begin{tcolorbox}[breakable, size=fbox, boxrule=1pt, pad at break*=1mm,colback=cellbackground, colframe=cellborder]
\prompt{In}{incolor}{ }{\boxspacing}
\begin{Verbatim}[commandchars=\\\{\}]
\PY{l+s+sd}{\PYZdq{}\PYZdq{}\PYZdq{}}

\PY{l+s+sd}{run = True}

\PY{l+s+sd}{while run:}
\PY{l+s+sd}{    screen.fill((255,255,255))}
\PY{l+s+sd}{    screen.blit(text, textRect)}

\PY{l+s+sd}{    for event in pygame.event.get():}
\PY{l+s+sd}{        if event.type == pygame.QUIT:}
\PY{l+s+sd}{            run = False}

\PY{l+s+sd}{        if event.type == pygame.KEYDOWN:}

\PY{l+s+sd}{            if event.key == pygame.K\PYZus{}RIGHT:}
\PY{l+s+sd}{                estado = trans[estado]}
\PY{l+s+sd}{                text = font.render(\PYZsq{}ESTADO \PYZsq{} + str(estado), True, (255,0,0), (255,255,255))}

\PY{l+s+sd}{            if event.key == pygame.K\PYZus{}LEFT:}
\PY{l+s+sd}{                if estado \PYZgt{}= 2:}
\PY{l+s+sd}{                    estado = estado \PYZhy{} 1}
\PY{l+s+sd}{                    text = font.render(\PYZsq{}ESTADO \PYZsq{} + str(estado), True, (255,0,0), (255,255,255))}

\PY{l+s+sd}{    draw\PYZus{}cells(estados[estado], N)}
\PY{l+s+sd}{    pygame.display.update()}
\PY{l+s+sd}{\PYZdq{}\PYZdq{}\PYZdq{}}
\end{Verbatim}
\end{tcolorbox}

    \hypertarget{simulauxe7uxe3o-automuxe1tica}{%
    \textbf{\large Simulação Automática}\label{simulauxe7uxe3o-automuxe1tica}}

    \begin{tcolorbox}[breakable, size=fbox, boxrule=1pt, pad at break*=1mm,colback=cellbackground, colframe=cellborder]
\prompt{In}{incolor}{ }{\boxspacing}
\begin{Verbatim}[commandchars=\\\{\}]
\PY{n}{run} \PY{o}{=} \PY{k+kc}{True}

\PY{n}{MOVEEVENT} \PY{o}{=} \PY{n}{pygame}\PY{o}{.}\PY{n}{USEREVENT}\PY{o}{+}\PY{l+m+mi}{1}
\PY{n}{T} \PY{o}{=} \PY{l+m+mi}{250}
\PY{n}{pygame}\PY{o}{.}\PY{n}{time}\PY{o}{.}\PY{n}{set\PYZus{}timer}\PY{p}{(}\PY{n}{MOVEEVENT}\PY{p}{,} \PY{n}{T}\PY{p}{)}
\PY{n}{estado} \PY{o}{=} \PY{l+m+mi}{1}

\PY{k}{while} \PY{n}{run}\PY{p}{:}
    \PY{n}{screen}\PY{o}{.}\PY{n}{fill}\PY{p}{(}\PY{p}{(}\PY{l+m+mi}{255}\PY{p}{,}\PY{l+m+mi}{255}\PY{p}{,}\PY{l+m+mi}{255}\PY{p}{)}\PY{p}{)}
    \PY{n}{screen}\PY{o}{.}\PY{n}{blit}\PY{p}{(}\PY{n}{text}\PY{p}{,} \PY{n}{textRect}\PY{p}{)}

    \PY{k}{for} \PY{n}{event} \PY{o+ow}{in} \PY{n}{pygame}\PY{o}{.}\PY{n}{event}\PY{o}{.}\PY{n}{get}\PY{p}{(}\PY{p}{)}\PY{p}{:}
        \PY{k}{if} \PY{n}{event}\PY{o}{.}\PY{n}{type} \PY{o}{==} \PY{n}{pygame}\PY{o}{.}\PY{n}{QUIT}\PY{p}{:}
            \PY{n}{run} \PY{o}{=} \PY{k+kc}{False}

        \PY{k}{if} \PY{n}{event}\PY{o}{.}\PY{n}{type} \PY{o}{==} \PY{n}{MOVEEVENT}\PY{p}{:}
            \PY{n}{estado} \PY{o}{=} \PY{n}{trans}\PY{p}{[}\PY{n}{estado}\PY{p}{]}
            \PY{n}{text} \PY{o}{=} \PY{n}{font}\PY{o}{.}\PY{n}{render}\PY{p}{(}\PY{l+s+s1}{\PYZsq{}}\PY{l+s+s1}{ESTADO }\PY{l+s+s1}{\PYZsq{}} \PY{o}{+} \PY{n+nb}{str}\PY{p}{(}\PY{n}{estado}\PY{p}{)}\PY{p}{,} \PY{k+kc}{True}\PY{p}{,} \PY{p}{(}\PY{l+m+mi}{255}\PY{p}{,}\PY{l+m+mi}{0}\PY{p}{,}\PY{l+m+mi}{0}\PY{p}{)}\PY{p}{,} \PY{p}{(}\PY{l+m+mi}{255}\PY{p}{,}\PY{l+m+mi}{255}\PY{p}{,}\PY{l+m+mi}{255}\PY{p}{)}\PY{p}{)}

    \PY{n}{draw\PYZus{}cells}\PY{p}{(}\PY{n}{estados}\PY{p}{[}\PY{n}{estado}\PY{p}{]}\PY{p}{,} \PY{n}{N}\PY{p}{)}
    \PY{n}{pygame}\PY{o}{.}\PY{n}{display}\PY{o}{.}\PY{n}{update}\PY{p}{(}\PY{p}{)}
\end{Verbatim}
\end{tcolorbox}

    \hypertarget{fim-do-pygame}{%
\textbf{\large Fim do pygame}\label{fim-do-pygame}}

    \begin{tcolorbox}[breakable, size=fbox, boxrule=1pt, pad at break*=1mm,colback=cellbackground, colframe=cellborder]
\prompt{In}{incolor}{ }{\boxspacing}
\begin{Verbatim}[commandchars=\\\{\}]
\PY{n}{pygame}\PY{o}{.}\PY{n}{quit}\PY{p}{(}\PY{p}{)}
\end{Verbatim}
\end{tcolorbox}


    % Add a bibliography block to the postdoc
    
    
    
\end{document}
